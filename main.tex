\documentclass[journal,onecolumn]{IEEEtran}

\usepackage{graphicx}
\usepackage{cite}
\usepackage{amsmath}
\usepackage{tabularx}
\usepackage{colortbl}
\usepackage{float}
\usepackage{upgreek}
\usepackage{siunitx}
\usepackage{booktabs}
\usepackage{subcaption}
\usepackage{verbatim}
\usepackage[hidelinks]{hyperref}
\usepackage[absolute,overlay]{textpos}
\usepackage[justification=centerlast]{caption}
\usepackage[utf8]{inputenc}
\usepackage[T1]{fontenc}
\usepackage[spanish]{babel}

% Columna elastica, rellena el espacio sobrante de la tabla,
% si permite saltos de linea
\newcolumntype{Y}{>{\centering\arraybackslash}X}

% Columna de ancho fijo pero permite saltos de linea
\newcolumntype{C}{>{\centering\arraybackslash}p{2cm}}
% Columna default c se ajusta automáticamente a la columna más ancha,
% no permite saltos de linea 

% Columna de ancho variable ejem: Z{3cm} pero centra verticalmente
\newcolumntype{Z}[1]{>{\centering\arraybackslash}m{#1}}

\definecolor{lightblue}{rgb}{0.8,0.85,1}
\definecolor{mediumblue}{rgb}{0.6,0.7,1}
\definecolor{darkblue}{rgb}{0.3,0.5,0.9}

\begin{document}

\begin{titlepage}

    \begin{center}
        \textbf{\LARGE INSTITUTO TECNOLÓGICO DE COSTA RICA}\\[1em]
        \textbf{\LARGE ESCUELA DE INGENIERÍA ELECTRÓNICA}\\[3em] 

        \begin{figure}[H]
            \centering
            \includegraphics[width=1\textwidth]{Images/Anexos/TEC.png}
        \end{figure}
        \vspace{2cm}
        
        \textbf{\LARGE \textquotedblleft Desarrollo de un sensor digital de irradiancia solar de bajo costo para aplicaciones de investigación y monitoreo ambiental\textquotedblright }\\[10em]

        \textbf{\LARGE Proyecto Final de Graduación para optar por el título de }\\[2em]
        
        \textbf{\LARGE Ingeniero Electrónico }\\[5em]

        \textbf{\LARGE con el grado académico de }\\[2em]
        
        \textbf{\LARGE Licenciatura}\\[10em]

        \textbf{\Large Adrián Rodríguez Murillo}\\[2em]

        \textbf{\Large San Carlos, Noviembre, 2025}
    \end{center}

\end{titlepage}

\newpage
\section*{\textbf{Hoja del Tribunal Evaluador}}
\addcontentsline{toc}{section}{Hoja del Tribunal Evaluador}
\begin{figure}[H]
    \hspace{-1cm}
    \includegraphics[width=1.1\textwidth]{Images/Anexos/Acta_Aprobación.png}
\end{figure}

\newpage
\section*{\textbf{Declaración de Autenticidad}}
\addcontentsline{toc}{section}{Declaración de Autenticidad}
\vspace{4cm}
Declaro que el presente Proyecto de Graduación ha sido realizado enteramente por mi persona, utilizando y aplicando literatura referente al tema e introduciendo conocimientos propios. \\

En los casos en que he utilizado bibliografía, he procedido a indicar las fuentes mediante las respectivas citas bibliográficas. \\

En consecuencia, asumo la responsabilidad total por el trabajo de graduación
realizado y por el contenido del correspondiente informe final. \\ \vspace{1cm}

Santa Clara, San Carlos, 24 de noviembre de 2025\\

Adrián Rodríguez Murillo \\

Céd: 207970960 \\

\begin{figure}[H]
    \hspace{11cm}
    \vspace{-0.5cm}
    \includegraphics[width=0.3\textwidth]{Images/Anexos/Firma_Adrian.png}
\end{figure}
\hspace{11.3cm} \textbf{\small Firmado con firma digital}

\newpage
\section*{\textbf{Resumen}}
\addcontentsline{toc}{section}{Resumen}

\section*{\textbf{Abstract}}
\addcontentsline{toc}{section}{Abstract}

% -------------------------------------------------------------------
% ÍNDICES
% -------------------------------------------------------------------

\newpage
\tableofcontents
\newpage
\listoffigures
\newpage
\listoftables


% =====================================================================
% SECCIÓN 1: INTRODUCCIÓN
% =====================================================================
\newpage
\section{\textbf{Introducción}}

Este capítulo presenta el contexto general de este proyecto, el problema que se desea abordar y la importancia de su solución. Asimismo, se introduce de forma general la solución desarrollada y su relevancia técnica, económica y ambiental. Se busca proporcionar al lector una visión integral del propósito, alcance e impacto del sistema propuesto.

\subsection{\textbf{Problema existente e importancia de su solución}}

Costa Rica posee condiciones geográficas y climáticas excepcionales para el aprovechamiento de energía solar como fuente de generación de energía eléctrica, situándose entre los países con mayor potencial del continente americano \cite{arrieta}. No obstante, esta ventaja competitiva no ha sido explotada adecuadamente: en 2024, solo el 0.20\% de la generación eléctrica renovable nacional provino de fuentes solares, mientras que el 64.70\% correspondió a recursos hídricos \cite{ice}. Esta marcada brecha entre potencial y aprovechamiento efectivo revela la necesidad urgente de fortalecer las capacidades tecnológicas nacionales en generación fotovoltaica. Sin embargo, previo al desarrollo y expansión de granjas solares, se deben abordar y fortalecer los pilares de investigación y monitoreo de los recursos solares, de manera que, se cuente de forma local con las herramientas para ejecutar planes de desarrollo planificados y ambientalmente sostenibles que maximicen a corto y largo plazo los beneficios de la inversión estatal y privada en tecnologías renovables fotovoltaicas.

\begin{figure}[H]
    \centering
    \includegraphics[width=0.8\textwidth]{Images/Gráficos/DOCSE_2024.png}
    \caption{Demanda energética por fuente en Costa Rica, año 2024 \cite{ice}.}
    \label{fig:Energy_Demand}
\end{figure}

Dentro del área de investigación y monitoreo de energías renovables fotovoltaicas, la instrumentación científica (piranómetros) para la medición de irradiancia solar global horizontal ($G$) \textit{-en adelante, se usará el término ``irradiancia'' para referirse a $G$ ya que no se utilizó ningún otro tipo de irradiancia o radiación durante la redacción de este trabajo, en caso de hacerlo se nombra explícitamente-} representa un componente esencial para el estudio, expansión y aprovechamiento de las capacidades energéticas nacionales. Los piranómetros de grado científico (Clase A) \textit{—basados en termopilas o fotodiodos calibrados—} actualmente se pueden adquirir por montos de entre \$1,500 y \$8,000 USD, mientras que los sensores más económicos del mercado (Clase C) \textit{-basados en células solares de referencia calibradas-} como los de la marca Spektron, oscilan entre \$115 y \$500 USD. Además, incluso los dispositivos más económicos presentan limitaciones de precisión, sensibilidad térmica y requerimientos de calibración periódica en laboratorios especializados, lo cual eleva los costos operativos totales y restringe las capacidades de investigación y desarrollo de los centros académicos o estatales frente a la industria privada \cite{dunlop}.

\begin{table}[H]
\caption{Comparación de sensores de irradiancia según su principio de funcionamiento}
\label{tabla_sensores_irradiancia}
\centering
\begin{tabularx}{\columnwidth}{|Y|Y|Z{3cm}|C|C|}
\hline
\rowcolor{darkblue} \textbf{Tipo de sensor} & \textbf{Ejemplo comercial} & \textbf{Rango espectral (nm)} & \textbf{Precisión} & \textbf{Costo (USD)} \\
\hline
\rowcolor{white} \textbf{Termopila} & Kipp \& Zonen CMP10 & 285–2800 & $\pm$2\% & 7,000–8,000 \\
\hline
\rowcolor{lightblue} \textbf{Fotodiodo de silicio} & Hukseflux SR11 & 300–1100 & $\pm$5\% & 1,200–1,800 \\
\hline
\rowcolor{white} \textbf{Célula solar de referencia} & Spektron 320 & \text{No especificado} & $\pm$5\% & 550–650 \\
\hline
\rowcolor{lightblue} \textbf{Sensor combinado (meteorológico)} & Davis 6450 Vantage Pro2 & 400–1100 & $\pm$8\% & 400–600 \\
\hline
\rowcolor{white} \textbf{Termopila} & Spektron SPN1 & 400–1100 & $\pm$10\% & 300–600 \\
\hline
\rowcolor{lightblue} \textbf{Fotodiodo calibrado} & Apogee SP-110 & 360–1120 & $\pm$5\% & 150–250 \\
\hline
\rowcolor{white} \textbf{Célula solar de referencia} & Spektron 210 & \text{No especificado} & $\pm$ 5\% & 150-200 \\ 
\hline
\end{tabularx}
\end{table}

Esta situación afecta especialmente a instituciones académicas y centros de investigación que enfrentan la necesidad de realizar mediciones de irradiancia precisas con presupuestos limitados. Como resultado, los procesos de investigación, validación experimental y desarrollo de prototipos fotovoltaicos suelen depender de equipos externos o de datos indirectos, reduciendo la capacidad nacional para innovar en tecnologías solares. \\

En este contexto, el Instituto Tecnológico de Costa Rica (TEC) se posiciona como un actor clave en el fortalecimiento del ecosistema científico nacional. La Escuela de Ingeniería Electrónica, a través del Laboratorio de Sistemas Electrónicos para la Sostenibilidad (SESlab), promueve el desarrollo de soluciones tecnológicas sostenibles, centradas en energías renovables, control automático e instrumentación de precisión. En este contexto, el SESlab proporciona un entorno ideal para desarrollar dispositivos de medición de irradiancia de bajo costo y alta precisión adaptados a la situación económica e industrial costarricense. \\

La idea de utilizar fotoceldas para medir irradiancia no es nueva. En 1998, King \cite{king} ya proponía diseños conceptuales de sistemas de medición, pero la tecnología disponible en ese momento solo permitía plasmar estas ideas en papel y realizar demostraciones en el laboratorio, sin la posibilidad de implementar proyectos funcionales. Hoy, gracias al avance de la microelectrónica, los sistemas embebidos y las comunicaciones inalámbricas, tecnologías que antes estaban restringidas a equipos de laboratorio de alto costo, se han vuelto accesibles a desarrolladores e investigadores. Plataformas como el ESP32, junto con sensores de corriente de precisión y módulos LoRa, permiten crear instrumentos de medición inteligentes, confiables y de bajo costo \cite{carrasco, cruz, orsetti, rhiat, risdiyanto}, haciendo realidad ideas que décadas atrás solo podían conceptualizarse. \\

La disponibilidad de instrumentación científica asequible genera múltiples beneficios:
\begin{itemize}
    \item Facilita la implementación de redes de monitoreo solar distribuidas a nivel nacional.
    \item Potencia la investigación y docencia en energías renovables y sistemas electrónicos.
    \item Promueve la soberanía tecnológica y la independencia en el desarrollo de hardware científico.
    \item Contribuye a la planificación energética mediante la obtención de datos de irradiancia locales.
    \item Impulsa la cooperación entre universidades, instituciones públicas y el sector industrial.
\end{itemize}

La relevancia de esta problemática no radica únicamente en el acceso económico a la instrumentación, sino en su potencial para fortalecer la formación de ingenieros capaces de diseñar, calibrar y optimizar sistemas científicos nacionales.

\subsection{\textbf{Solución seleccionada}}

El proyecto plantea el desarrollo de un sensor digital de irradiancia de bajo costo basado en células fotovoltaicas comerciales como transductores primarios. Este enfoque se fundamenta en la relación directa entre la irradiancia incidente y la corriente de cortocircuito ($I_{SC}$) generada por una célula solar \cite{bliss, ibrahim}. A diferencia de los sistemas basados en termopilas, esta técnica permite implementar sensores precisos, económicos y fácilmente replicables. \\

La solución desarrollada está compuesta por los siguientes subsistemas:

\begin{itemize}
    \item \textit{Subsistema de transducción:} Emplea células solares monocristalinas KXOB25-14X1F-TR, seleccionadas por su buena disponibilidad comercial y amplia sensitividad espectral, así como una hoja de datos del fabricante bien definida que permite realizar modelos teóricos para evaluar su comportamiento en pruebas de campo.
    \item \textit{Subsistema de acondicionamiento de señal:} Utiliza sensores/amplificadores de precisión INA226 para medir $I_{SC}$ y convertidores analógico-digitales ADS1115 de 16 bits para medir temperatura ($T$), proporcionando resolución suficiente para variaciones de $G < 1 W/m^2$ .
    \item \textit{Subsistema de compensación térmica:} Incorpora termistores USP10982, que permiten compensar la dependencia térmica en la \( I_{SC} \) mediante un modelo lineal utilizando el coeficiente de variación térmica reportado por el fabricante.
    \item \textit{Subsistema de Procesamiento y Comunicación:} Basado en el módulo TTGO ESP32 LoRa32 V2.1, con conectividad WiFi para cargar datos a una plataforma de análisis de datos IoT como Thinkspeak, antena LoRa para transmisión remota de datos y pantalla OLED para visualizar datos durante tareas de mantenimiento y calibración.
    \item \textit{Subsistema de Almacenamiento:} Utiliza una tarjeta microSD en la ranura integrada del TTGO, garantizando redundancia y minimizando las perdidas de datos.
\end{itemize}

\begin{figure}[H]
    \centering
    \includegraphics[width=0.8\textwidth]{Images/Diagramas/Diagrama_de_Bloques.png}
    \caption{Diagrama de bloques general del sistema propuesto. Fuente: Elaboración propia}
    \label{fig:System_Block_Diagram}
\end{figure}

La comunicación inalámbrica por protocolo Low Power Long Range (LP LoRa) permite la integración del sistema como un nodo IoT dentro de la red de sensores distribuida del SESlab, reportando datos a una estación base (Gateway) con un alcance superior a 5 km en condiciones de línea de vista.

El encapsulado utiliza una caja de plexo de exteriores con estándares IP65 modificada, asegurando resistencia ambiental y durabilidad en condiciones de campo, así como ventilación y protección ante insectos. Este diseño prioriza la modularidad y repetibilidad mediante el uso exclusivo de componentes disponibles comercialmente, favoreciendo su escalabilidad y evolución temporal en iteraciones subsecuentes.

En síntesis, la solución propuesta busca combinar accesibilidad económica, precisión metrológica y sostenibilidad ambiental, promoviendo el desarrollo de capacidades locales en instrumentación solar y fortaleciendo la infraestructura científica nacional.



% =====================================================================
% SECCIÓN 2: META Y OBJETIVOS
% =====================================================================
\newpage
\section{\textbf{Meta y Objetivos}}

\subsection{\textbf{Meta}}

Desarrollar un prototipo funcional de sensor digital de irradiancia de bajo costo que democratice el acceso a tecnologías de medición solar en instituciones educativas y proyectos de investigación con recursos limitados, estableciendo una alternativa viable y técnicamente competente a los instrumentos comerciales de alto costo actualmente disponibles en el mercado, y contribuyendo al fortalecimiento de capacidades locales en desarrollo de instrumentación científica para aplicaciones de energías renovables.

\subsection{\textbf{Objetivo general}}

Desarrollar y validar un sistema de medición de irradiancia digital, de bajo costo y alta precisión, basado en tecnologías de comunicación inalámbrica de largo alcance, que permita la implementación de redes de monitoreo solarimétrico distribuidas para aplicaciones educativas, de investigación y monitoreo ambiental.

\textit{Indicador de cumplimiento:} Sensor digital activo con precisión mayor o igual al 95\% respecto a sensores comerciales calibrados, con costo de producción inferior a \$70 USD por unidad.

\subsection{\textbf{Objetivos específicos}}

\begin{enumerate}
    \item Diseñar un circuito de adquisición y acondicionamiento de señal de alta precisión para la medición de $I_{SC}$, capaz de detectar variaciones de $G < 1 W/m^2$ en el rango operativo de $0-1500 W/m^2$.
    
    \textit{Indicador de cumplimiento:} Documentación del diseño detallado del sistema, incluyendo diagramas de bloques, diagramas de flujo, esquemáticos eléctricos, diseño de PCB y lista de materiales. \\

    \item Desarrollar el ensamblaje físico del sistema, que garantice funcionalidad, protección contra humedad, altas temperaturas, condiciones meteorológicas adversas y protección contra insectos y animales pequeños.
    
    \textit{Indicador de cumplimiento:} Ensamble físico del proyecto listo para instalación y operación en exteriores por periodos de tiempo prolongados. \\

    \item Implementar un algoritmo de adquisición y procesamiento de datos que permita conversión, compensación térmica, calibración, almacenamiento local y remoto, visualización y transmisión inalámbrica de datos.

    \textit{Indicador de cumplimiento:} Código fuente documentado y funcional que implemente todas las funcionalidades especificadas, con capacidad de operación autónoma como nodo IoT mediante tecnología LoRa. \\

    \item Realizar el proceso de calibración por regresión lineal, cuantificando la precisión, variabilidad y linealidad de la respuesta del sistema respecto a una referencia.

    \textit{Indicador de cumplimiento:} Informe de calibración que documente los coeficientes de corrección y demás variables estadísticas que demuestren la precisión alcanzada por el sistema. \\

    \item Ejecutar pruebas de validación de campo durante un período mínimo de 15 días, bajo diversas condiciones atmosféricas que demuestren el correcto funcionamiento del sistema durante periodos prolongados de tiempo.

    \textit{Indicador de cumplimiento:} Base de datos generada a partir de las mediciones de campo, con sus respectivas representaciones gráficas y análisis estadísticos de los datos. \\

    \item Elaborar documentación técnica que incluya repositorios, procedimientos de calibración, manuales de operación y mantenimiento y análisis de costos, facilitando la replicación del diseño por parte de otras organizaciones.

    \textit{Indicador de cumplimiento:} Repositorio de Github con el código del sistema y repositorio de Google Drive con los archivos del proyecto que incluya documentación final del proyecto en formato pdf, archivos fuente, esquemáticos, layout, hojas de datos, referencias, manuales, lista de componentes y demás archivos necesarios para la correcta replicación del proyecto.
\end{enumerate}

% =====================================================================
% SECCIÓN 3: MARCO TEÓRICO
% =====================================================================
\newpage
\section{\textbf{Marco Teórico}}

\subsection{\textbf{Descripción del sistema o proceso a mejorar}}
En la actualidad, los sistemas comerciales de medición de irradiancia han alcanzado niveles altos de precisión. Sin embargo, también son altos los costos de adquisición, calibración y mantenimiento de dichos sistemas, lo cual representa un desafío para su adopción masiva en proyectos que buscan optimizar recursos de generación solar. Por ello, el presente proyecto se enfoca en el estudio y desarrollo de una técnica que permita reducir los costos de fabricación de sensores de irradiancia sin comprometer la precisión de las mediciones, buscando así una solución más accesible y eficiente para organizaciones con recursos económicos limitados. \\

Se planteó un sistema de medición de irradiancia solar de bajo costo orientado a su replicabilidad dentro de entornos IoT y operación a largo plazo. El sistema se estructuró con un elemento de transducción, acondicionamiento analógico y conversión A/D de alta resolución, cómputo embebido y comunicaciones de baja potencia, con registro local y envío a plataforma en la nube. La arquitectura está diseñada para operar en exteriores, con énfasis en robustez, trazabilidad metrológica y escalabilidad hacia una red de nodos IoT. El diagrama de flujo general del proceso se representó en la Fig.~\ref{fig:diagrama_flujo}, desde la adquisición física hasta la visualización y análisis en plataforma remota.

\begin{figure}[H]
    \centering
    \includegraphics[width=0.67\textwidth]{Images/Diagramas/Diagrama_de_Flujo.png}
    \caption{Diagrama de flujo del sistema de medición de irradiancia. Fuente: Elaboración propia}
    \label{fig:diagrama_flujo}
\end{figure}

Las etapas principales del sistema son:
\begin{enumerate}
    \item \textit{Adquisición de datos:} Medición de $I_{SC}$ y $T$ de cada sensor.
    \item \textit{Filtrado de señales:} Mitigación de ruido mediante desacopladores y técnicas de acondicionamiento como media movil.
    \item \textit{Compensación térmica:} Ajuste de lecturas para corregir el efecto de la variación de $T$ sobre las células solares.
    \item \textit{Procesamiento local:} Cálculo de $G$ corregida y calibrada, almacenamiento local y preparación de paquetes de transmisión.
    \item \textit{Transmisión inalámbrica LoRa:} Envío de datos mediante LoRa hacia el Gateway central.
    \item \textit{Transmisión inalámbrica WiFi:} Envío de datos mediante solicitud HTTP hacia la nube de Thinkspeak. \\
\end{enumerate}

El sistema se insertó dentro de un entorno IoT de monitoreo solar perteneciente al SESlab, constituido por nodos de medición distribuidos de distintas variables tanto eléctricas como abióticas de instalaciones de generación fotovoltaica del TEC Sede San Carlos y un Gateway basado en Raspberry Pi 4 con conexión a una base de datos, instalado en el Laboratorio PROTEC de la misma sede.

\subsection{\textbf{Antecedentes bibliográficos}}

En años recientes se produjo un volumen considerable de investigación sobre radiación solar, tecnología fotovoltaica (FV), metrología de irradiancia y soporte embebido/IoT. Dada la magnitud y el alcance de la investigación en esta área, este apartado no pretendió ser un estado del arte exhaustivo; más bien, se organizó suficiente evidencia reciente por categorías para sustentar las decisiones de diseño, calibración y validación del trabajo. A continuación se resumieron los principales cuerpos de conocimiento y sus hallazgos relevantes.

\begin{itemize}
    \item \textit{Teoría de radiación solar y métricas:} Se consolidaron definiciones de irradiancia/irradiación, espectro AM1.5 y criterios de medición, así como referencias a magnitudes absolutas de la irradiancia solar total y modelado de la variabilidad geofísica \cite{kopp, dominguez, voyant, wald}. Estas fuentes sustentaron la selección de ventanas de muestreo y la interpretación de variabilidad (segundos–minutos) usada más adelante en filtrado y anti–alias. También se establecieron valores límite teóricos para poder validar las mediciones de campo.

    \item \textit{Principios FV y dependencia de $I_{sc}$ con $G$ y $T$:} Se revisaron fundamentos de conversión FV, efecto fotoeléctrico y dependencias térmicas/espectrales de la respuesta \cite{marques, al-ezzi, jho, dubey, liu, khan, venkateswari}. En particular, se emplearon resultados sobre la linealidad aproximada de $I_{sc}$ con la irradiancia y el coeficiente térmico de $I_{sc}$ para formular el modelo de calibración y compensación con término de temperatura \cite{li}.

    \item \textit{Modelado de dispositivos FV:} Se adoptaron modelos de un diodo y metodologías de identificación de parámetros para describir la relación $I$–$V$ y su extrapolación a condiciones fuera de laboratorio \cite{tobon, garcía, vinod, obeidat}. Estos trabajos permitieron acotar supuestos cuando se operó la célula en cortocircuito como transductor de irradiancia.

    \item \textit{Medición de irradiancia, estándares y clases ISO~9060:} Se documentó el uso de piranómetros de termopila y su clasificación (clases A/B/C) según ISO~9060:2018, con incertidumbres de referencia y costos asociados \cite{kipp}. Este marco normativo se utilizó como referencia solarimétrica para comparar desempeño/deriva de soluciones de bajo costo y para diseñar esquemas de comparación.

    \item \textit{Calibración y corrección metrológica:} Se tomaron guías industriales, estudios de linealidad para definir el procedimiento de calibración y métricas ($R^2$, $RMSE$, $MAPE$, $GUM$, $MBE$) para evaluar las estimaciones de irradiancia \cite{balanzategui,ordoñez, dunlop, bliss, li}. El uso de $I_{sc}$ como variable diagnóstica no destructiva reforzó la elección de la célula de Si como elemento de transducción \cite{ibrahim}.

    \item \textit{Sensores de bajo costo basados en célula de referencia:} Se recopilaron diseños y validaciones que emplearon células FV como sensor de irradiancia, mostrando correlaciones casi lineales con errores post–calibración típicos en el rango 3–8~\% \cite{king, carrasco, cruz, orsetti, rhiat, risdiyanto}. También se hallaron implementaciones que integraron temperatura y registro local como parte del ``stack" de medición \cite{ogundimu}.

    \item \textit{Sensores basados en fotodiodo:} Se consideraron alternativas con fotodiodo/transimpedancia, destacando su rapidez y control espectral, y propuestas con estimadores asistidos por modelos/ML para compensar dependencias \cite{ogundimu, chouay}. Estas líneas justificaron la evaluación paralela de ambas familias (célula de Si y fotodiodo).

    \item \textit{Estimación y pronóstico con imágenes del cielo y métodos de aprendizaje automático:} Se integraron antecedentes sobre sky-imagers, visión por computadora y modelos modernos (incluidos transformadores) para estimar o pronosticar componentes de irradiancia y su variabilidad estocástica \cite{altaani, ansong, mercier, sanchez, roy}. Aunque el enfoque de este trabajo fue sensado físico local, estas técnicas aportaron contexto sobre escalas de variabilidad y potencial de fusión de fuentes.

    \item \textit{Instrumentación embebida, conversión y buses:} Se consultaron referencias técnicas para la selección de ADCs $\Delta\Sigma$ y monitores de corriente por shunt (ADS1115, INA226), así como prácticas de diseño (PGA, ENOB, filtrado, desacoples) \cite{kester, rehpade, guangzhao}. Las hojas de datos de los componentes principales ofrecieron especificaciones eléctricas/espectrales relevantes \cite{ixolartm_datasheet, ina226_datasheet}.

    \item \textit{Comunicaciones LPWAN (LoRa/LoRaWAN) y operación regional:} Se resumieron parámetros físicos (SF, BW, CR, CRC), sensibilidad y control de tasa, junto con perfiles/reglas regionales (US915) pertinentes para Costa Rica \cite{lorawan, sx1276}. Estos antecedentes guiaron el compromiso entre robustez del enlace y consumo.

\end{itemize}

\vspace{2mm}
\noindent
Como síntesis operativa, la Tabla~\ref{tab:mapa_antecedentes} mapeó categorías, preguntas guía y contribuciones clave que se trasladaron al diseño y validación del prototipo.

\begin{table}[H]
\caption{Mapa de antecedentes: categorías, foco y aportes al diseño}
\label{tab:mapa_antecedentes}
\centering
\begin{tabularx}{\columnwidth}{|Y|Y|Y|}
\hline
\rowcolor{darkblue}\textbf{Categoría} & \textbf{Pregunta guía} & \textbf{Fuentes y aporte clave} \\
\hline
\rowcolor{white}Radiación y métricas & ¿Qué se midió y a qué escala temporal/espectral? & \cite{kipp, kopp, dominguez, voyant}: Definiciones, espectro AM1.5, variabilidad (seg–min). \\
\hline
\rowcolor{lightblue}Principios FV \& $I_{sc}$ & ¿Cómo se relacionó la señal con $G$ y $T$? & \cite{marques, al-ezzi, jho, dubey, li}: Linealidad de $I_{sc}$ y coeficiente térmico. \\
\hline
\rowcolor{white}Modelado FV & ¿Qué modelo físico respaldó supuestos de operación? & \cite{tobon, garcía, vinod, obeidat}: Modelo 1–diodo. \\
\hline
\rowcolor{lightblue}ISO~9060 y referencia & ¿Cuál fue la vara metrológica? & \cite{kipp, hukseflux_buy}: Clases A/B/C y criterios de desempeño. \\
\hline
\rowcolor{white}Calibración \& incertidumbre & ¿Cómo se ajustó y verificó la exactitud? & \cite{dunlop, bliss, ibrahim, li}: Linealidad, $R^2$, $RMSE$, $MAPE$, $GUM$, $MBE$. \\
\hline
\rowcolor{lightblue}Célula de referencia & ¿Qué ofrece el enfoque de bajo costo? & \cite{king, carrasco, cruz, orsetti, rhiat, risdiyanto}: Error 3–8~\% tras calibración. \\
\hline
\rowcolor{white}Fotodiodo \& ML & ¿Alternativas y compensaciones? & \cite{ogundimu, chouay}: Rapidez, TIA, estimadores de corrección. \\
\hline
\rowcolor{lightblue}Embebido \& medición & ¿Cómo se garantizó resolución/robustez? & \cite{kester, ina226_datasheet, ixolartm_datasheet}: ENOB, shunt, prácticas de diseño. \\
\hline
\rowcolor{white}LPWAN \& normativa & ¿Cómo se aseguró el enlace de campo? & \cite{lorawan, sx1276}: SF/BW/CR,CRC, US915 y ADR. \\
\hline
\end{tabularx}
\end{table}

\noindent
Este esquema por categorías permitió enlazar la teoría y las prácticas solarimétricas con decisiones concretas de transducción, acondicionamiento, conversión A/D, calibración, comunicación y registro que se detallaron en las secciones siguientes.

\subsection{\textbf{Fundamentos teóricos}}

\subsubsection{Radiación Solar, Irradiancia e Irradiación}

La radiación solar que incide en la parte superior de la atmósfera terrestre (conocida como el espectro AM0) abarca un amplio rango de longitudes de onda del espectro electromagnético, desde los rayos X de alta energía hasta las ondas de radio. Sin embargo, la mayor parte de la energía se concentra en un rango más acotado, principalmente entre los 150 y los 4000 nm. Dentro de este rango, la energía se distribuye de manera aproximada en un 7\% de radiación ultravioleta ($UV \approx [150-400] nm$), un 44\% de luz visible ($VL \approx [400-800] nm$) y un 49\% de radiación infrarroja ($IR \approx [800-4000] nm$). El pico de intensidad de esta radiación ocurre cerca de los 500 nm, en la región del espectro visible \cite{wald}. \\

\begin{figure}[H]
    \centering
    \includegraphics[width=0.8\textwidth]{Images/Gráficos/Solar_Espectrum_1.png}
    \caption{Espectro de radiación solar acotado a las componentes más contributivas \cite{kopp}.}
    \label{fig:Espectro_solar_1}
\end{figure}

La irradiancia (\(G\)) se define como el flujo de potencia radiante que incide por unidad de área, expresado en [\(W/m^{2}\)]. Al integrar esta potencia a lo largo del tiempo, se obtiene la irradiación (\(H\)), que se mide en [\(J/m^{2}\)] o [\(Wh/m^{2}\)]. En la superficie terrestre, la irradiancia se describe frecuentemente como la suma de sus componentes directa y difusa, a lo que se denomina irradiancia global en plano horizontal (\(GHI\)). Este espectro solar terrestre se aproxima mediante las condiciones estándar (\(AM1.5G\)), un modelo empleado para la caracterización de dispositivos fotovoltaicos y sensores \cite{dunlop, parida}. Este valor en la superficie es significativamente menor que la potencia solar total (\(TSI\)) en la cima de la atmósfera, la cual se sitúa alrededor de \(1361~W/m^{2}\) con variaciones anuales mínimas \cite{kopp}.

\begin{figure}[H]
    \centering
    \includegraphics[width=0.68\textwidth]{Images/Diagramas/Elementos_de_irradiancia.png}
    \caption{ Elementos de la irradiancia solar \cite{dominguez}.}
    \label{fig:componentes_radiacion}
\end{figure}

Al transitar a través de la atmósfera terrestre, la radiación solar es modificada por procesos de absorción y dispersión que alteran su distribución espectral de forma selectiva. Esta absorción molecular ocurre porque gases como el ozono (\(O_{3}\)), el vapor de agua (\(H_{2}O\)), el dióxido de carbono (\(CO_{2}\)), y otros componentes atmosféricos, tienen bandas de absorción específicas en distintas longitudes de onda. Por ejemplo, el ozono en la estratosfera es el principal responsable de absorber la radiación ultravioleta (UV) de alta energía (por debajo de ~300 nm), protegiendo la superficie de sus efectos nocivos. El vapor de agua y el dióxido de carbono, por su parte, absorben fuertemente en la región infrarroja, creando "ventanas atmosféricas" donde la radiación de otras longitudes de onda puede transmitirse. Además de la absorción, la dispersión por moléculas de aire (dispersión de Rayleigh) y partículas más grandes (dispersión de Mie) desvía los fotones en todas direcciones, lo que reduce aún más la irradiancia directa y contribuye a la componente difusa de la irradiancia global. El resultado es una curva espectral en la superficie que es significativamente diferente de la que se mide en el espacio (AM0) \cite{wald}.

\begin{figure}[H]
    \centering
    \includegraphics[width=0.8\textwidth]{Images/Gráficos/Solar_Espectrum_2.png}
    \caption{Espectro de radiación solar, comparación entre diferentes distribuciones atmosféricas \cite{wald}.}
    \label{fig:Espectro_solar_2}
\end{figure}

\subsubsection{Importancia de la medición de irradiancia solar}

La medición precisa de la irradiancia solar constituye un pilar esencial para diversas aplicaciones en meteorología, agricultura de precisión, arquitectura sostenible y sistemas de energía renovable. Su cuantificación precisa es crucial para el dimensionamiento de sistemas fotovoltaicos y la predicción energética \cite{kopp}.

Tradicionalmente, la medición se realiza mediante piranómetros de termopila o sensores basados en fotodetectores semiconductores. Los piranómetros de termopila se consideran estándar de referencia según la Organización Meteorológica Mundial (WMO) \cite{kipp} y presentan ventajas en estabilidad temporal y respuesta espectral plana en el rango 300-2800 nm. Sin embargo, su costo y complejidad de fabricación limitan su uso en aplicaciones distribuidas o proyectos de bajo presupuesto.

En contraste, sensores basados en células solares o fotodiodos ofrecen construcción simple, respuesta rápida y costos significativamente menores, pero requieren técnicas de compensación térmica y calibración cuidadosa para alcanzar precisión comparable a los piranómetros de referencia \cite{roy}. \\

\subsubsection{Medición de irradiancia}

En esta sección se presentan las principales alternativas a los piranómetros de grado científico utilizadas para la medición de irradiancia solar, destacando sus principios de funcionamiento, características generales, ventajas y desventajas. Es fundamental presentar esta comparativa para que las decisiones de ingeniería se tomen con bases sólidas sobre alternativas tecnológicas disponibles y se pueda garantizar una solución que cumpla y a su vez genere un equilibrio óptimo para los requisitos de confiabilidad, linealidad, complejidad, costos, facilidad de integración y precisión de los datos en aplicaciones fotovoltaicas y estudios solarimétricos \cite{dunlop, king, carrasco}.

\paragraph{Células solares}

Los sensores basados en células solares funcionan aprovechando el efecto fotoeléctrico para generar una corriente de cortocircuito proporcional a la irradiancia incidente \cite{jho, tobon}.

\textit{Características generales:}  
\begin{itemize}
    \item Bajo costo y fácil disponibilidad comercial.  
    \item Linealidad en el rango de operación normal de la célula.  
    \item Fácil integración con microcontroladores y sistemas embebidos.  
\end{itemize}

\textit{Ventajas:}  
\begin{itemize}
    \item No requieren circuitos complejos.  
    \item Pueden implementarse en arreglos distribuidos para monitoreo local de irradiancia.  
\end{itemize}

\textit{Desventajas:}  
\begin{itemize}
    \item La precisión depende de la calibración y de la temperatura ambiente.  
    \item Sensibles a degradación por exposición prolongada a la radiación UV.  
\end{itemize}

\begin{figure}[H]
    \centering
    \includegraphics[width=0.78\textwidth]{Images/Ejemplos/Arentio.png}
    \caption{Sensor de irradiancia basado en célula solar de referencia, Arentio \cite{arentio_buy}.}
    \label{fig:celda_solar}
\end{figure}

\paragraph{Fotodiodos}

Los fotodiodos generan una corriente proporcional a la intensidad de luz incidente, permitiendo mediciones rápidas de irradiancia \cite{king, ogundimu}. Su respuesta espectral puede adaptarse mediante filtros ópticos para simular la respuesta de radiación solar estándar.

\textit{Características generales:}  
\begin{itemize}
    \item Alta velocidad de respuesta, adecuada para mediciones de corta duración o fluctuaciones rápidas.  
    \item Sensibilidad espectral ajustable mediante filtros.  
\end{itemize}

\textit{Ventajas:}  
\begin{itemize}
    \item Permiten mediciones con alta frecuencia de muestreo.  
    \item Tamaño compacto y bajo consumo de energía.  
\end{itemize}

\textit{Desventajas:}  
\begin{itemize}
    \item Requieren circuitos de amplificación y filtrado para reducir ruido.  
    \item Linealidad limitada frente a cambios de intensidad muy altos.  
\end{itemize}

\begin{figure}[H]
    \centering
    \includegraphics[width=0.8\textwidth]{Images/Ejemplos/SR20-D2.png}
    \caption{Sensor de irradiancia basado en fotodiodo, Hukseflux
Digital Class A pyranometer SR20-D2 \cite{hukseflux_buy}.}
    \label{fig:fotodiodo}
\end{figure}

\paragraph{Sky-imagers}

Los sky-imagers utilizan cámaras y procesamiento de imágenes del cielo para estimar la irradiancia difusa y directa \cite{ansong, mercier, sanchez}. Son particularmente útiles para predicciones de generación solar y estudios meteorológicos.

\textit{Características generales:}  
\begin{itemize}
    \item Analizan la cobertura nubosa y la radiación solar incidente mediante algoritmos de visión computacional.  
    \item Permiten estimaciones de irradiancia en diferentes condiciones atmosféricas.  
\end{itemize}

\textit{Ventajas:}  
\begin{itemize}
    \item Capaces de estimar irradiancia difusa y directa simultáneamente.  
    \item Útiles en predicciones de corto plazo de generación fotovoltaica.  
\end{itemize}

\textit{Desventajas:}  
\begin{itemize}
    \item Mayor complejidad en hardware y procesamiento de datos.  
    \item Requieren calibración periódica de la cámara y algoritmos sofisticados.  
\end{itemize}

\begin{figure}[H]
    \centering
    \includegraphics[width=0.6\textwidth]{Images/Ejemplos/Sky-imager.png}
    \caption{Sistema óptico ASI desarrollado y ubicado en el Centro de Investigaciones en Óptica-Aguascalientes \cite{sanchez}.}
    \label{fig:sky_imager}
\end{figure}

\subsubsection{Efecto fotoeléctrico}

Las células solares convierten la radiación electromagnética ($EMR$) proveniente del sol en energía eléctrica a través del efecto fotoeléctrico \cite{jho, al-ezzi, marques}. Este fenómeno físico ocurre cuando los fotones incidentes sobre un material semiconductor transfieren su energía a los electrones del material, liberándolos de sus enlaces atómicos y generando pares electrón-hueco. Estos portadores de carga son posteriormente separados por el campo eléctrico interno del dispositivo, produciendo una corriente eléctrica continua (corriente fotovoltaica) en el circuito externo.

El efecto fotoeléctrico solo ocurre cuando los fotones de la luz incidente tienen una energía suficiente para superar la energía de enlace de los electrones del material. Esta energía mínima se conoce como la función de trabajo en metales y, en el caso de los semiconductores, está relacionada con el ancho de la banda prohibida (bandgap).

La energía de un fotón (\(E\)) está directamente relacionada con su frecuencia (\(f\)) a través de la ecuación de Planck-Einstein: 

\begin{equation}
  E=hf  
\end{equation}

donde $h$ es la constante de Planck. Si la energía del fotón es mayor que el bandgap del semiconductor, el fotón es absorbido, y su energía se transfiere a un electrón, que es promovido desde la banda de valencia a la banda de conducción, creando un par electrón-hueco. Los fotones con energía inferior al bandgap no son absorbidos y simplemente pasan a través del material.

La estructura fundamental para lograr la separación de los portadores de carga es la unión p-n, una región formada por la unión de un semiconductor tipo p (con exceso de huecos) y un semiconductor tipo n (con exceso de electrones). En la unión p-n, se forma una región de agotamiento con un campo eléctrico interno incorporado. Cuando la luz incide en la célula y crea pares electrón-hueco cerca de esta unión, el campo eléctrico separa rápidamente a los portadores de carga, los electrones libres son arrastrados hacia el lado n y los huecos son atraídos hacia el lado p. Esta separación evita que el electrón y el hueco se recombinen, lo que anularía el efecto fotovoltaico. 

\begin{figure}[H]
    \centering
    \includegraphics[width=0.76\textwidth]{Images/Gráficos/Efecto_fotoeléctrico.png}
    \caption{Efecto fotoeléctrico ilustrado en la junta n-p de una célula solar \cite{al-ezzi}.}
    \label{fig:photoelectric_efect}
\end{figure}

La separación efectiva de estos portadores es crucial para la eficiencia de la célula. Al separar los electrones y los huecos en lados opuestos de la unión p-n, se establece una diferencia de potencial eléctrico (voltaje) a través de la célula solar, similar a la que se produce en una batería. Cuando la célula se conecta a una carga externa (por ejemplo, una bombilla o un circuito), los electrones fluyen desde el lado n, a través del circuito, hacia el lado p para recombinarse con los huecos. Este flujo de electrones constituye la corriente eléctrica continua que alimenta la carga. \\

\subsubsection{Relación entre señal del sensor e irradiancia}

En una célula de silicio operada en cortocircuito, la irradiancia $G$ se aproximó como proporcional a la corriente $I_{SC}$:

\begin{equation}
G \,\propto\, \alpha_{CAL} *  k_G * \frac{I_{SC}(G, T)}{1 + \alpha_{T}(T_{med} - T_{ref})} + \beta_{CAL},
\label{eq:IscG}
\end{equation}

donde $\alpha_{CAL}$ y $\beta_{CAL}$ representan los componentes de pendiente y offset calculados por regresión lineal para calibrar el sistema con respecto a la referencia, $k_G$ es una constante efectiva de sensibilidad ($(W/m^2)/A$) y $\alpha_T$ es el coeficiente térmico relativo ($\%/^\circ C$), ambos deducidos de las condiciones de caracterización y valores teóricos de las células en la hoja de datos del fabricante \textit{-razón relevante para utilizar células de fabricantes confiables-}, mientras que $T_{med}$, $T_{ref}$ son las temperaturas en grados Celsius medidas y de referencia respectivamente. Además, es importante tener en cuenta que la proporcionalidad teórica de los resultados se puede degradar por:

\begin{itemize}
    \item \textit{Mismatch espectral:} La $G$ medida depende del solapamiento entre el espectro incidente $S(\lambda) \approx [300-2000] nm$ \cite{wald} y la responsividad del sensor $R_S(\lambda) \approx [300-1200] nm$ \cite{ixolartm_datasheet}, por lo tanto, se ``filtra'' una parte de la radiación incidente, de manera que $R_S(\lambda)$ representa aproximadamente el $85-90\%$ de la energía contenida en $S(\lambda)$.
    \item \textit{Respuesta angular:} En sensores planos sin cúpula difusora, especialmente al amanecer y atardecer, debido al ángulo de incidencia de la luz se genera un efecto de reflexión y se ``pierde'' una parte de la potencia, ese error angular residual debe ser incorporado al presupuesto de incertidumbre.
    \item \textit{Temperatura:} En un dispositivo semiconductor, el aumento de la temperatura superficial reduce el ancho de la banda prohibida (bandgap), permitiendo que fotones de menor energía sean absorbidos y, por tanto, incrementando la tasa de generación de pares electrón-hueco. Esta mayor generación de portadores de carga se traduce en un ligero aumento en la corriente de cortocircuito (\(I_{SC}\)).
\end{itemize}

Para abordar la incertidumbre, limitaciones y desviaciones intrínsecas del sistema, se incorporaron los términos $\alpha_{CAL}$, $k_G$, $\alpha_T$, $\beta_{CAL}$ y se realizaron los procesos de calibración con una referencia confiable.

Diversos estudios han demostrado que, si se implementan técnicas de calibración y compensación térmica adecuadas, los sensores basados en células solares comerciales de bajo costo pueden alcanzar una precisión comparable a la de los piranómetros estándar. Esto permite su uso en aplicaciones de monitoreo de radiación solar distribuida, estudios de eficiencia energética de paneles fotovoltaicos y experimentos académicos de bajo presupuesto \cite{carrasco, rhiat, cruz}. \\

\subsubsection{Compensación térmica en celdas solares}

Las variaciones de temperatura alteran la corriente de cortocircuito ($I_{SC}$) y el voltaje de circuito abierto ($V_{OC}$) en células solares de manera opuesta. En el caso específico de las células IXOLAR™ SolarBIT KXOB25-14X1F-TR de la marca ANYSOLAR, se reporta un coeficiente de temperatura de cortocircuito positivo $\Delta I_{SC}/ \Delta T = 26.5 uA/K$ y un coeficiente de temperatura de voltaje de circuito abierto negativo $\Delta V_{OC}/ \Delta T = -1.74 mV/K$ \cite{ixolartm_datasheet}. La corrección térmica es, por lo tanto, esencial para normalizar las mediciones a una temperatura de referencia, típicamente $25 ^\circ C$, que es el estándar de la industria fotovoltaica. Para lograr esto, se aplica una compensación a las mediciones de $I_{SC}$ basada en el coeficiente de temperatura de la corriente. Esta corrección se aplica mediante la siguiente fórmula:  

\begin{equation}
I_{sc,corr} = \frac{I_{sc,med}}{1 + \alpha_{T}(T_{med} - T_{ref})}
\end{equation}

donde $T_{med}$ es la temperatura medida de la célula, $T_{ref}$ la temperatura de referencia ($25 ^\circ C$) y $\alpha_T$ es el coeficiente de temperatura de cortocircuito relativo [$\%/^\circ C$]. La correcta implementación de esta compensación permite mantener errores inferiores al 2\% en el rango de operación de 50-1500 W/m² \cite{dubey}.

El rendimiento de las células fotovoltaicas es fuertemente dependiente de la temperatura, ya que esta modifica las propiedades eléctricas de los semiconductores. Aunque la irradiancia es el principal factor que determina la potencia de salida, la temperatura de operación de la célula puede tener un impacto significativo y debe ser corregida para obtener mediciones precisas y comparables. Este efecto es especialmente relevante en aplicaciones de monitoreo y calibración, donde se buscan resultados fiables bajo diferentes condiciones ambientales.

\begin{figure}[H]
    \centering
    \includegraphics[width=0.7\textwidth]{Images/Gráficos/Temp_vs_Irrad.png}
    \caption{Efecto de la temperatura sobre irradiancia/corriente de una célula solar de silicio \cite{dubey}.}
    \label{fig:temp_vs_irrad}
\end{figure}

\subsubsection{Modelado de celdas solares}

El comportamiento de una célula solar se modela frecuentemente mediante el modelo de un solo diodo ($SDM$ por sus siglas en ingles), que representa el funcionamiento de la unión p-n y las imperfecciones internas del dispositivo. Este modelo es ampliamente utilizado debido a su buen equilibrio entre precisión y simplicidad, ya que permite estimar y predecir las características eléctricas (curva I-V) de una célula o módulo fotovoltaico bajo diversas condiciones de operación, como la irradiancia y la temperatura.

El modelo de un solo diodo representa una célula solar con los siguientes componentes en un circuito equivalente:

\begin{itemize}
    \item \textit{Fuente de corriente constante ($I_{ph}$):} Simula la corriente generada por el efecto fotoeléctrico al incidir la luz solar sobre la célula.
    \item \textit{Diodo ideal:} Representa la unión p-n del semiconductor, que solo permite el flujo de corriente en una dirección. La corriente a través del diodo ($I_D$) se modela con la ecuación de Shockley.
    \item \textit{Resistencia en serie ($R_s$):} Representa las pérdidas resistivas debidas al flujo de corriente a través de la red de contacto metálico, el material semiconductor y las interconexiones. Una alta resistencia en serie reduce la corriente y la tensión de salida.
    \item \textit{Resistencia en paralelo ($R_{sh}$):} Representa las pérdidas de corriente causadas por imperfecciones en la unión p-n o por otros defectos. Una baja resistencia en paralelo provoca fugas de corriente, lo que reduce la eficiencia.
\end{itemize}

\begin{figure}[H]
    \centering
    \includegraphics[width=0.7\textwidth]{Images/Diagramas/One_diode_model.png}
    \caption{Modelo de un diodo para panel solar \cite{marques}.}
    \label{fig:one_diode_model}
\end{figure}

La ecuación que describe la relación entre la corriente ($I$) y el voltaje ($V$) en este circuito equivalente se basa en la ley de corrientes de Kirchhoff y es la siguiente:

\begin{equation}
    I=I_{ph}-I_{D}-I_{sh}  
\end{equation}

Sustituyendo los modelos para el diodo y la resistencia de derivación ($R_{sh}$), la ecuación completa del modelo de un solo diodo es:

\begin{equation}
I = I_{ph} - I_0 \left( e^{\frac{V + I R_s}{n V_t}} - 1 \right) - \frac{V + I R_s}{R_{sh}}
\end{equation}

donde \cite{tobon, vinod, garcía}:

\begin{itemize}
    \item $I$ es la corriente total de la célula [A].
    \item $I_{ph}$ es la corriente fotovoltaica generada por la luz [A].
    \item $I_0$ es la corriente de saturación inversa del diodo [A].
    \item $V$ es el voltaje aplicado [V].
    \item $R_s$ es la resistencia serie de la célula [\(\Omega\)].
    \item $R_{sh}$ es la resistencia shunt de la célula [\(\Omega\)].
    \item $n$ es el factor de idealidad del diodo.
    \item $V_t$ es el voltaje térmico, definido como $kT/q$ [V].
    \item $q$ es la carga del electrón [Coulombs].
    \item $k$ es la constante de Boltzmann [J/K].
    \item $T$ es la temperatura absoluta [K]. \\
\end{itemize}

La resolución de esta ecuación, que relaciona la corriente con el voltaje a través de una función implícita, permite generar la curva I-V característica de la célula solar. Esta curva es fundamental para determinar los puntos de operación óptimos y para analizar el rendimiento del dispositivo bajo diferentes condiciones.

\begin{figure}[H]
    \centering
    \includegraphics[width=0.8\textwidth]{Images/Gráficos/Curva_I-V.png}
    \caption{Ejemplo de curva I-V para diferentes valores de $R_{sh}$ \cite{marques}.}
    \label{fig:curve_i-v}
\end{figure}

La curva I-V que se grafica a partir de esta ecuación muestra la misma forma característica que una curva medida experimentalmente, con una región de corriente constante seguida de una caída exponencial en el voltaje. \\

\subsubsection{Calibración y trazabilidad}

Para modelar la relación entre el instrumento bajo calibración y el patrón de referencia, se emplean diversas técnicas de ajuste de curva, siendo los métodos de regresión estadística los más comunes. La regresión lineal se utiliza cuando se asume una relación lineal entre las lecturas del sensor y las del instrumento de referencia \cite{balanzategui}. Este método, basado en el ajuste de una línea recta que minimiza el error cuadrático medio, es simple y efectivo para dispositivos con una respuesta proporcional y predecible. Sin embargo, para aquellos sensores con comportamientos no lineales o en rangos de medición más amplios, una regresión polinomial puede ofrecer un ajuste más preciso. Al extender el modelo lineal con términos de orden superior (cuadráticos, cúbicos, etc.), la regresión polinomial es capaz de capturar curvaturas y desviaciones más complejas en la respuesta del sensor. La selección del grado del polinomio es crucial para evitar el sobreajuste (overfitting), un fenómeno donde el modelo se ajusta demasiado a los datos de calibración y pierde capacidad de generalización. La validación estadística de estos modelos, como el análisis de residuos y los indicadores de ajuste, es esencial para garantizar la fiabilidad del proceso de calibración.

Considerando la naturaleza intrínseca de los dispositivos fotovoltaicos, la regresión lineal se presenta como la técnica de calibración más adecuada para sensores basados en células solares \cite{dunlop}. Esto se debe a que la corriente de cortocircuito ($I_{SC}$) de una célula fotovoltaica es, en primera aproximación, directamente proporcional a la irradiancia incidente. Aunque otros factores como la temperatura y el espectro solar pueden influir, el modelo lineal captura con gran precisión la relación fundamental entre la intensidad de la luz y la respuesta del sensor. Esta relación de proporcionalidad simple justifica el uso de un modelo lineal para estimar la irradiancia a partir de la $I_{SC}$ medida, ofreciendo una solución robusta, interpretable y computacionalmente eficiente para la calibración de sensores de bajo costo.

\begin{figure}[H]
    \centering
    \includegraphics[width=0.8\textwidth]{Images/Pre-Calibración/SolarBit_1_Calibration.png}
    \caption{Ejemplo de regresión lineal para calibración Fuente: Elaboración propia.}
    \label{fig:linear_regression}
\end{figure}

Durante la calibración y validación de un modelo de regresión, es crucial emplear métricas estadísticas para cuantificar su rendimiento y fiabilidad. El coeficiente de determinación, $R^{2}$, evalúa la proporción de la varianza en la variable dependiente que puede ser explicada por la variable independiente, con valores cercanos a 1 indicando un ajuste más robusto del modelo a los datos. Para una cuantificación del error, se utilizan el Error Cuadrático Medio ($RMSE$) y el Error Porcentual Absoluto Medio ($MAPE$). El $RMSE$ mide la magnitud promedio de los errores en las unidades de la variable dependiente, penalizando más los errores grandes al elevarlos al cuadrado. El $MAPE$, por su parte, expresa el error como un porcentaje promedio, facilitando su interpretación en términos relativos. El Error de Sesgo Medio ($MBE$) indica la dirección del sesgo promedio, donde un valor cercano a cero sugiere que no hay sobreestimación ni subestimación sistemática en las predicciones. Finalmente, la Guía para la Expresión de la Incertidumbre en la Medición ($GUM$) proporciona un marco metodológico internacional para evaluar y reportar la incertidumbre asociada a la calibración, garantizando la trazabilidad y comparabilidad de los resultados en un contexto metrológico. En conjunto, estas métricas permiten no solo ajustar el modelo de regresión de manera óptima, sino también validar su capacidad predictiva y reportar su precisión de manera estandarizada y completa. \\

\newpage

\subsection{\textbf{Fundamentos técnicos}}

\subsubsection{Tecnologías de medición de corriente}

Para medir la corriente de cortocircuito ($I_{sc}$) con precisión en sensores basados en células solares, se pueden emplear distintas técnicas, cada una con compromisos entre costo, precisión y complejidad del diseño. La elección del método depende principalmente del rango de corriente, la linealidad deseada y las limitaciones del sistema de adquisición de datos.

\paragraph{Resistencias shunt} Consiste en colocar una resistencia de precisión de bajo valor en serie con la célula solar, de modo que la caída de tensión a través de ella sea proporcional a la corriente según la ley de Ohm. Este voltaje puede ser amplificado mediante un amplificador operacional y digitalizado posteriormente.
\begin{itemize}
    \item Ventajas: Simplicidad de implementación, bajo costo y buena linealidad en un amplio rango de corriente.
    \item Desventajas: Disipación de potencia proporcional al cuadrado de la corriente y posible alteración del punto de operación de la célula (distanciamiento del punto real de cortocircuito).
\end{itemize}

\paragraph{Amplificadores de sensado de corriente integrados} Estos dispositivos combinan en un solo board comercial la resistencia shunt, un amplificador de instrumentación y un conversor analógico-digital (ADC) de alta resolución. Permiten una medición precisa con bajo nivel de ruido, y algunos modelos integran además cálculo de potencia o energía, como los sensores de la familia INA2xx.
\begin{itemize}
    \item Ventajas: Alta precisión, bajo nivel de ruido, reducción de componentes externos y comunicación digital directa con microcontroladores.
    \item Desventajas: Mayor costo relativo y limitación en el rango de corriente medible que depende de la resistencia de shunt instalada.
\end{itemize}

\paragraph{Sensores de efecto Hall} Emplean el principio del efecto Hall para detectar el campo magnético generado por el flujo de corriente. Permiten medir corriente sin contacto eléctrico directo, lo cual evita perturbaciones en el circuito.
\begin{itemize}
    \item Ventajas: Aislamiento galvánico completo, adecuada respuesta dinámica y medición tanto de corriente continua como alterna.
    \item Desventajas: Menor precisión y resolución frente a técnicas resistivas, además de requerir calibración térmica y electrónica adicional.
\end{itemize}

\paragraph{Amplificadores de transimpedancia} Se emplean principalmente cuando la célula solar opera como fotodiodo o fuente de corriente en modo de alta impedancia. El amplificador convierte directamente la corriente generada en un voltaje proporcional, con mínima carga sobre la fuente.
\begin{itemize}
    \item Ventajas: Alta sensibilidad, mínima perturbación de la fuente de corriente y respuesta rápida.
    \item Desventajas: Mayor susceptibilidad al ruido térmico y necesidad de componentes de precisión en el lazo de realimentación. \\
\end{itemize}

\subsubsection{Tecnologías de medición de temperatura}

En instrumentación ambiental y solarimétrica existe necesidad de medir temperatura para correcciones de sensibilidad, evaluación de deriva y trazabilidad de condiciones \cite{hukseflux_buy}. En términos generales, se distinguieron cinco familias tecnológicas ampliamente disponibles en el mercado. A continuación se definieron sus principios, rangos típicos y principales ventajas y desventajas.

\paragraph{Termistores (NTC/PTC)} 
Los termistores son resistores cuyo valor varía fuertemente con la temperatura. En termistores NTC la resistencia decrece con $T$, mientras que en los PTC aumenta. 

\begin{itemize}
    \item \textit{Ventajas:} Muy bajo costo; elevada sensibilidad en el rango meteorológico habitual; tamaño y masa térmica reducidos.
    \item \textit{Desventajas:} No linealidad intrínseca; deriva por envejecimiento; autocalentamiento si la excitación no fue mínima; exactitud dependiente de la curva y de la calibración local \cite{kester}.
\end{itemize}

Para termistores NTC frecuentemente se utilizan dos modelos de linealización: Modelo Steinhart–Hart (tri–parámetro) y la aproximación $\beta$ (bi–parámetro). La forma completa de Steinhart–Hart exige conocer tres coeficientes $(A,B,C)$ \textit{—normalmente obtenidos a partir de tres puntos de calibración o provistos por el fabricante—} y se expresa como:
\begin{equation}
    \frac{1}{T} \;=\; A + B \ln R + C (\ln R)^3,    
\end{equation}

con $T$ en kelvin y $R$ la resistencia del termistor. Este modelo ofrece baja desviación en un rango amplio, pero requiere datos que no siempre están disponibles en hojas técnicas de uso general \cite{li}. Por ello, en aplicaciones prácticas y literatura aplicada se adopta con frecuencia la aproximación $\beta$ por su simplicidad y menor requerimiento de parámetros.

La fórmula $\beta$ se deriva del comportamiento exponencial de $R(T)$ y necesita únicamente un par de referencias $(R_0, T_0)$ \textit{$R_0 = \text{Valor de resistencia del termistor a la temperatura de referencia}$ y $T_0=25 ^\circ C$ la temperatura de referencia—} y la constante $\beta$ de la hoja de datos del fabricante:
\begin{equation}
    T \;=\; \left(\frac{1}{T_0} + \frac{1}{\beta}\,\ln\!\frac{R(T)}{R_0}\right)^{-1}.    
\end{equation}

Donde $R(T)$ es la resistencia medida.

Esta segunda aproximación mantiene el error de linealización bajo en torno al punto operativo de referencia ($T_0$), a cambio, sacrifica exactitud en zonas térmicas alejadas. En la práctica, su uso resulta conveniente en rangos meteorológicos habituales [$0-60 ^\circ C$] siempre que se controlen otras fuentes de error como el autocalentamiento por corriente de excitación, gradientes de contacto o tolerancia del NTC \cite{li}. 

\paragraph{Detectores de resistencia metálica (RTD)}
Se definen como sensores basados en la variación casi lineal de la resistividad metálica con $T$ \textit{-coeficiente $\alpha$ cercano a $0.00385 ^\circ C^{-1}$ para platino industrial-}.
\begin{itemize}
    \item \textit{Ventajas:} Alta estabilidad a largo plazo; excelente repetibilidad y linealidad en el rango [$-40-150 ^\circ C$]; adecuada como referencia interna de instrumentos de clase alta \cite{hukseflux_buy}.
    \item \textbf{Desventajas:} Mayor costo y complejidad de interfaz (puente, amplificación diferencial, compensación de cables); masa térmica mayor que la de un NTC de chip \cite{li}.
\end{itemize}

\paragraph{Termopares (tipos K/T)} 
El termopar es una unión de dos metales distintos que generan una fuerza termoeléctrica proporcional a la diferencia de temperatura entre la unión de medida y la de referencia.
\begin{itemize}
    \item \textbf{Ventajas:} Rango muy amplio (hasta centenas de $^\circ C$); sonda robusta; respuesta rápida con hilos finos.
    \item \textbf{Desventajas:} Sensibilidad por grado relativamente baja; necesidad de compensación precisa de la unión fría; mayor susceptibilidad a errores por gradientes en cables y uniones \cite{li}.
\end{itemize}

\paragraph{Pirómetros infrarrojos} 
Son sensores que estiman la temperatura superficial a partir de la radiación infrarroja emitida. Destaca la dependencia con la emisividad y el campo de visión.
\begin{itemize}
    \item \textbf{Ventajas:} Medición sin contacto; tiempos de respuesta muy breves; útil cuando el acceso físico resulta difícil o indeseable.
    \item \textbf{Desventajas:} Necesidad de conocer/ajustar emisividad; sensibilidad a radiación externa y reflejos; resultados fuertemente dependientes de la geometría y distancia; menos apropiado para correcciones metrológicas finas sin control riguroso de condiciones. \\
\end{itemize}

\subsubsection{Tecnologías de filtrado de ruido}

El ruido en sistemas de adquisición puede provenir de fuentes internas (alimentaciones, conmutación de microcontroladores, transmisión de alta frecuencia) o externas como interferencias electromagnéticas. Para minimizar estos comportamientos indeseados existen diversas técnicas:

\paragraph{Filtrado en dominio de frecuencia} Uso de filtros pasa-bajas, pasa-altas, pasa-banda o notch para atenuar componentes de frecuencia específicas que se sabe que son ruido. Por ejemplo, un filtro pasa-bajas elimina el ruido de alta frecuencia, mientras que un notch puede eliminar una frecuencia particular (por ejemplo, de línea de alimentación).
    
\paragraph{Promedio móvil (Moving Average)} Consiste en tomar una ventana de valores recientes, calcular su media y usar ese valor como salida suavizada. Es simple, eficaz contra fluctuaciones rápidas aleatorias, aunque introduce retardo.

\paragraph{Filtros adaptativos} Utilización de algoritmos que ajustan sus parámetros dinámicamente según las características de la señal y del ruido. Un ejemplo es el filtro de correlación adaptativa, que puede usar una señal de referencia de ruido para eliminar interferencias desconocidas o variables en el tiempo.

\paragraph{Descomposición empírica en modos intrínsecos ($EMD$ por sus siglas en inglés)} Técnica no lineal y basada en datos que divide la señal en “modos” oscilatorios intrínsecos, permitiendo separar tendencias, ruido, artefactos, especialmente útil cuando el ruido está en bandas frecuenciales que se solapan con la señal útil.

\paragraph{Filtrado modal en PCBs / Diseño físico} Métodos que involucran la configuración física, como layout de placa, rutas de trazado, técnicas de diseño para reducir acoplamientos (crosstalk), uso de filtros comunes/diferenciales, apantallamientos, etc., que pueden prevenir que el ruido ingrese al sistema.

\paragraph{Capacitores de desacople y capacitores bulk} En el dominio del hardware, el uso de capacitores de desacople cerca de las entradas de alimentación de los circuitos integrados reduce el ruido de alta frecuencia generado por conmutación, mientras que los capacitores bulk de mayor capacidad estabilizan el voltaje frente a caídas transitorias en la línea de alimentación. Esta combinación permite mantener una referencia de voltaje estable y minimizar la propagación del ruido a través de la fuente de alimentación \cite{rehpade, guangzhao}. \\

\subsubsection{Tecnologías de transmisión inalámbrica}

En sistemas de monitoreo IoT, la elección del protocolo de comunicación depende del compromiso entre alcance, tasa de datos y consumo energético. Entre las tecnologías más empleadas destacan Wi-Fi, Bluetooth Low Energy (BLE), ZigBee y LoRa, cada una optimizada para distintos escenarios de aplicación.
\begin{itemize}
    \item \textit{Wi-Fi:} ofrece alta tasa de transferencia (decenas de Mbps) pero elevado consumo energético, adecuado para estaciones con alimentación continua.
    \item \textit{BLE y ZigBee:} se orientan a redes personales o de corto alcance, con bajo consumo y topología mallada. 
    \item \textit{LoRa:} se distingue por su gran alcance y consumo ultrabajo, ideal para sistemas remotos alimentados por energía solar.
\end{itemize}

\paragraph{Low Power Long Range (LP LoRa)}LoRa es una tecnología de modulación basada en Chirp Spread Spectrum ($CSS$), que codifica los datos mediante variaciones en la frecuencia instantánea de una señal portadora. Este método proporciona robustez frente al ruido y permite comunicaciones de largo alcance, alcanzando hasta $15 km$ en condiciones de línea de vista, con velocidades que varían entre $0.3 - 50 Kbps$. 

\paragraph{LoRaWAN}La arquitectura LoRaWAN (Long Range Wide Area Network) define la capa superior del protocolo, gestionando la comunicación entre los nodos finales, gateways y servidores de red. Los dispositivos se clasifican en tres clases según su patrón de comunicación y consumo energético:  
\begin{itemize}
    \item \textit{Clase A:} Comunicación iniciada por el dispositivo, con mínima energía y mayor latencia.
    \item \textit{Clase B:} Incluye ventanas de recepción programadas, mejorando la sincronización.
    \item \textit{Clase C:} Permite recepción continua, a costa de mayor consumo energético. \\
\end{itemize}

\subsubsection{Arquitectura híbrida Edge–Cloud}

El modelo de procesamiento híbrido \textit{Edge–Cloud} representa una evolución en la arquitectura de sistemas IoT, combinando el procesamiento local de datos con la capacidad analítica de la computación en la nube. Esta integración busca equilibrar los requerimientos de latencia, ancho de banda y autonomía energética presentes en sistemas distribuidos de adquisición de datos.

Desde un punto de vista conceptual, la computación en el borde (\textit{Edge Computing}) se refiere al procesamiento de información cerca de la fuente de generación de datos, reduciendo la necesidad de transmisión continua hacia servidores remotos. Esto permite ejecutar tareas como filtrado digital, compensación térmica o detección de anomalías directamente en el nodo sensor, disminuyendo la dependencia de la conectividad y mejorando la resiliencia del sistema.

Por otro lado, la computación en la nube (\textit{Cloud Computing}) provee infraestructura escalable para el almacenamiento histórico, análisis avanzado y visualización de los datos recolectados. Esta capa ofrece capacidades superiores de procesamiento y algoritmos de inteligencia artificial que pueden ser utilizados para identificar patrones temporales o evaluar la eficiencia energética del sistema fotovoltaico.

La combinación de ambos enfoques da origen a una arquitectura híbrida, donde cada nivel cumple una función específica: 
\begin{itemize}
    \item \textit{Capa Edge:} Realiza preprocesamiento, control y almacenamiento temporal de datos en el dispositivo embebido.
    \item \textit{Capa Gateway:} Coordina la comunicación entre múltiples nodos, agregando y sincronizando la información.
    \item \textit{Capa Cloud:} Ejecuta análisis globales, mantiene bases de datos históricas y gestiona la interfaz con el usuario final.
\end{itemize}

Este esquema distribuido es especialmente relevante en redes de monitoreo solarimétrico, donde la variabilidad ambiental y la densidad de sensores requieren un equilibrio entre procesamiento local y centralizado. Así, el paradigma \textit{Edge–Cloud} contribuye a una gestión más eficiente de los recursos computacionales y a la reducción del consumo energético total del sistema. \\

\subsubsection{Procesamiento embebido para sistemas IoT}

El procesamiento embebido constituye la base de los sistemas modernos de adquisición y monitoreo, permitiendo la ejecución de tareas específicas directamente en el dispositivo de medición. A diferencia de los sistemas de cómputo de propósito general, los sistemas embebidos integran hardware y software en una misma plataforma optimizada para realizar funciones concretas con restricciones de consumo, tamaño y costo.

En el contexto de aplicaciones IoT (Internet of Things), los microcontroladores con capacidades de conectividad inalámbrica \textit{—como WiFi, Bluetooth o LoRa—} permiten la creación de nodos inteligentes capaces de procesar, filtrar y transmitir información de manera autónoma. Este enfoque descentralizado reduce la carga en los servidores centrales y mejora la eficiencia energética de la red.

La familia de microcontroladores ESP32 es un ejemplo representativo de esta tendencia. Estos dispositivos incorporan una arquitectura de 32 bits con procesadores de alto rendimiento y periféricos integrados para comunicación digital, lo que posibilita el desarrollo de sistemas de medición distribuidos y de bajo costo. Desde el punto de vista teórico, su funcionamiento se fundamenta en el principio de procesamiento local, donde las señales analógicas son digitalizadas mediante convertidores ADC y tratadas por rutinas embebidas antes de ser enviadas a la red.

El paradigma del \textit{edge computing} complementa esta filosofía, al situar parte del procesamiento de datos cerca de la fuente de información. En sistemas de monitoreo solar, por ejemplo, los microcontroladores embebidos pueden calcular promedios móviles, filtrar ruido o detectar fallos antes de transmitir los datos al servidor, optimizando así el ancho de banda y la confiabilidad del sistema. Este modelo de procesamiento distribuido es esencial para lograr escalabilidad y autonomía energética en redes IoT dedicadas a la medición ambiental o energética. \\

\newpage

% =====================================================================
% SECCIÓN 4: PROCEDIMIENTO METODOLÓGICO
% =====================================================================

\section{Procedimiento Metodológico}

\subsection{Reconocimiento y definición del problema}

La fase inicial del proyecto se enfocó en la identificación precisa de las necesidades y limitaciones existentes en materia de instrumentación para la medición de irradiancia solar en el contexto de instituciones académicas y centros de investigación con recursos limitados.

El presente proyecto está inmerso dentro del marco de investigación y desarrollo del laboratorio SESlab del Instituto Tecnológico de Costa Rica. Por esta razón, toda decisión clave y/o crítica de desarrollo fue considerada, evaluada y aprobada por el equipo de investigadores del SESlab.

La investigación del estado del arte reveló que la solución al problema planteado tiene múltiples posibles caminos de desarrollo, de manera que los criterios principales para definir los componentes, técnicas y metodologías que se usaron en la solución final fueron los siguientes:
\begin{itemize}
    \item \textbf{Precio:} El principal criterio de diseño del proyecto responde directamente al objetivo de desarrollar un sistema de medición de irradiancia confiable y de bajo costo para democratizar el acceso a esta tecnología.
    \item \textbf{Compatibilidad intermodular:} Existen dependencias tecnológicas que garantizan el correcto desempeño del sistema, por ejemplo; debe existir compatibilidad entre el transductor solar primario seleccionado y la técnica de medición de $I_{SC}$, de manera que se asegure la integridad y confiabilidad de las mediciones.
    \item \textbf{Compatibilidad IoT:} Este criterio condiciona el módulo de comunicación seleccionado, debido a que ya existe un gateway funcional basado en Raspberry Pi 4 y antenas LoRa que maneja la recolección y almacenamiento en bases de datos de los distintos nodos descentralizados de medición, por lo tanto, cualquier nuevo nodo que se incorpore a la red debe ser compatible con los elementos que actualmente la componen.
    \item \textbf{Replicabilidad y escalabilidad:} El SESlab tiene interes en aumentar la cantidad de nodos del sistema IoT de medición de variables agrivoltáicas y solarimétricas, por lo tanto se debe garantizar la reproducibilidad del sistema y sus resultados.
\end{itemize}

\subsection{Análisis de necesidades}

Se realizaron sesiones de consulta con los investigadores del SESlab para identificar los requerimientos específicos de la instrumentación solarimétrica que se planteó como objetivo. Las necesidades identificadas incluyeron:

\begin{itemize}
    \item Capacidad de integración en redes de monitoreo distribuidas con múltiples nodos de medición.
    \item Acceso remoto a los datos en tiempo real para análisis y visualización.
    \item Almacenamiento local de los datos para garantizar redundancia y minimizar la perdida de información.
    \item Precisión comparable con instrumentos comerciales de investigación en energías renovables.
    \item Costo de adquisición compatible con presupuestos académicos limitados.
    \item Facilidad de replicación y mantenimiento con recursos locales.
\end{itemize}

\subsection{Definición de especificaciones objetivo}

Con base en el análisis de necesidades y la revisión de soluciones existentes, se establecieron las especificaciones objetivo del sistema a desarrollar:

\subsubsection{Especificaciones de desempeño}
\begin{itemize}
    \item Rango de medición: 0-1500 W/m².
    \item Precisión: $\geq$95\% comparado con instrumentos de referencia.
    \item Resolución: $\leq$1 W/m².
    \item Compensación térmica: -10°C a 65°C.
    \item Tiempo de respuesta: $<$15 segundos.
    \item Estabilidad temporal: deriva $<$2\% anual.
\end{itemize}

\subsubsection{Especificaciones funcionales}
\begin{itemize}
    \item Almacenamiento local: $>$10 años de lecturas.
    \item Transmisión inalámbrica: alcance $>$500 m (línea de vista).
    \item Interfaz local: pantalla OLED para visualización inmediata.
    \item Protección ambiental: IP65 o similar.
\end{itemize}

\subsubsection{Restricciones de diseño}
\begin{itemize}
    \item Costo total de materiales: $<$\$70 USD.
    \item Componentes comercialmente disponibles.
    \item Herramientas de desarrollo de código abierto.
    \item Documentación completa para replicabilidad.
\end{itemize}



\subsection{Análisis de componentes disponibles}

Se realizó un estudio de mercado de componentes electrónicos disponibles comercialmente, evaluando:

\begin{itemize}
    \item Células solares de diferentes tecnologías (monocristalina, policristalina, amorfa).
    \item Sensores de corriente de alta precisión.
    \item Sensores de temperatura de respuesta rápida.
    \item Conversores ADC de precisión.
    \item Elementos de filtrado de ruido.
    \item Microcontroladores con capacidades de comunicación inalámbrica.
    \item Módulos de comunicación compatibles con LoRa.
    \item Sistemas de almacenamiento de datos locales.
    \item Sistema de manejo y almacenaje de datos en la nube.
    \item Componentes de protección y acondicionamiento de señal.
\end{itemize}

Para cada categoría de componentes se evaluaron criterios de: disponibilidad en el mercado local e internacional, costos de adquisición incluyendo envíos, documentación técnica disponible, soporte comunitario y ejemplos de implementación.

\subsection{Evaluación de alternativas}

Las siguientes secciones detallan las opciones evaluadas, con comparativas tipo tabla de Pugh, finalizando con una sección que resume y justifica la selección de cada componente clave del sistema.

\subsubsection{Célula Solar}
La célula solar es uno de los componentes clave del diseño; como elemento principal del sistema, debe cumplir con criterios de precisión, costo, accesibilidad y durabilidad para que el proyecto sea viable a largo plazo.

\begin{table}[H]
\caption{Evaluación de celdas solares}
\label{tabla_pugh_celdas_solares}
\centering
\begin{tabularx}{\columnwidth}{|Y|Y|Y|Y|Y|}
\hline
\rowcolor{darkblue} \textbf{Criterio} & \textbf{	
SunPower Maxeon Gen III Je3A} & \textbf{KXOB25-14X1F-TR
} & \textbf{SUNYIMA Células Monocristalinas} & \textbf{IXYS SLMD121H08L} \\
\hline
\rowcolor{white} Linealidad respuesta & 2 & 2 & 0 & 2 \\
\hline
\rowcolor{lightblue} Estabilidad temporal & 1 & 1 & 0 & 1 \\
\hline
\rowcolor{white} Costo & 1 & 2 & 2 & 1 \\
\hline
\rowcolor{lightblue} Disponibilidad & 1 & 1 & 2 & 1 \\
\hline
\rowcolor{white} Documentación & 0 & 2 & -1 & 2 \\
\hline
\rowcolor{darkblue} Total & 5 & 8 & 3 & 7 \\
\hline
\end{tabularx}
\end{table}

\paragraph{Justificación de la selección}
La célula solar KXOB25-14X1F-TR de la marca Anysolar presenta características superiores para aplicaciones de medición. Su respuesta espectral optimizada y coeficiente de temperatura bien documentado facilitan la implementación de algoritmos de compensación precisos \cite{ixolartm_datasheet}. La corriente de cortocircuito proporciona un nivel de señal adecuado para el rango de medición objetivo. En general, los criterios decisivos fueron costo y documentación, esto último influyendo fuertemente en la decisión final, puesto que facilita el modelado y estimación de datos, permitiendo a su vez realizar análisis precisos de desempeño.

\subsubsection{Sensor de corriente}

La medición precisa de la $I_{sc}$ es fundamental para la estimación confiable de la irradiancia solar. Para este propósito se analizaron distintas técnicas de sensado, considerando su precisión, complejidad, costo y adecuación a sistemas embebidos de bajo consumo. La Tabla \ref{tabla_pugh_sensores_corriente} presenta la comparación de cuatro alternativas representativas.

\begin{table}[H]
\caption{Evaluación de sensores de corriente}
\label{tabla_pugh_sensores_corriente}
\centering
\begin{tabularx}{\columnwidth}{|Y|Y|Y|Y|Y|}
\hline
\rowcolor{darkblue} \textbf{Criterio} & \textbf{Resistencia Shunt + Amplificador} & \textbf{Sensor de Efecto Hall} & \textbf{Amplificador de Transimpedancia} & \textbf{INA226 (Integrado)} \\
\hline
\rowcolor{white} Resolución / Linealidad & 1 & 0 & 2 & 2 \\
\hline
\rowcolor{lightblue} Precisión de corriente & 1 & 0 & 2 & 2 \\
\hline
\rowcolor{white} Consumo energético & 2 & 1 & 1 & 1 \\
\hline
\rowcolor{lightblue} Costo & 2 & 0 & 1 & 1 \\
\hline
\rowcolor{white} Facilidad de integración & 1 & 1 & 0 & 2 \\
\hline
\rowcolor{lightblue} Aislamiento galvánico & 0 & 2 & 0 & 0 \\
\hline
\rowcolor{darkblue} Total & 7 & 4 & 6 & 8 \\
\hline
\end{tabularx}
\end{table}

\paragraph{Justificación de la selección}

De acuerdo con los resultados obtenidos en la matriz de decisión, el INA226 representa la mejor alternativa para el sensado de corriente en este sistema. Este dispositivo ofrece un equilibrio óptimo entre precisión, resolución y facilidad de integración, superando las limitaciones de las configuraciones discretas.
El INA226 integra un amplificador de instrumentación diferencial, una resistencia shunt de precisión y un convertidor ADC de 16 bits, permitiendo mediciones directas de corriente y tensión con bajo nivel de ruido \cite{ina226_datasheet}. Su comunicación digital mediante protocolo I²C simplifica la conexión con microcontroladores como el ESP32, reduciendo la cantidad de componentes externos y el error asociado al cableado analógico. 

\subsubsection{Sensor de temperatura}

La compensación térmica del sistema solarimétrico requiere sensores de temperatura con buena precisión, baja inercia térmica y estabilidad a largo plazo, instalados en proximidad directa con la célula solar de referencia. La medición precisa de la temperatura superficial de la célula permite corregir la variación de la $I_{sc}$ respecto a la temperatura, mejorando la exactitud en la estimación de la irradiancia.

\begin{table}[H]
\caption{Evaluación de sensores de temperatura}
\label{tabla_pugh_sensores_temperatura}
\centering
\begin{tabularx}{\columnwidth}{|c|Y|Y|Y|Y|Y|}
\hline
\rowcolor{darkblue} \textbf{Criterio} & \textbf{Thin-film NTC} & \textbf{USP10982} & \textbf{DS18B20} & \textbf{LM35DZ} & \textbf{DHT22} \\
\hline
\rowcolor{white} Precisión & 1 & 2 & 2 & 0 & -1 \\
\hline
\rowcolor{lightblue} Tiempo de respuesta & 2 & 2 & 0 & 1 & -1 \\
\hline
\rowcolor{white} Consumo energético & 2 & 2 & 1 & 0 & -1 \\
\hline
\rowcolor{lightblue} Costo & 2 & 1 & 0 & 1 & 0 \\
\hline
\rowcolor{white} Facilidad de integración & 1 & 1 & 2 & 2 & 2 \\
\hline
\rowcolor{lightblue} Resistencia a la intemperie & 0 & 2 & 2 & 1 & 0 \\
\hline
\rowcolor{darkblue} Total & 8 & 10 & 7 & 5 & -1 \\
\hline
\end{tabularx}
\end{table}

\paragraph{Justificación de la selección}

De acuerdo con los resultados de la evaluación, los sensores basados en termistores NTC, y en particular el modelo USP10982, ofrecen la mejor combinación de desempeño térmico, precisión y simplicidad de integración. Estos dispositivos destacan por su bajo costo, rápida respuesta ante variaciones térmicas y encapsulados adecuados para mediciones a la intemperie.

\subsubsection{Convertidor Analógico-Digital}

La medición de la temperatura mediante termistores requiere conversores analógico-digital con precisión suficiente para distinguir pequeñas variaciones de voltaje en los divisores resistivos. La resolución, linealidad y bajo ruido son criterios fundamentales, mientras que el costo y consumo energético determinan la viabilidad de implementación.

\begin{table}[H]
\caption{Evaluación de convertidores ADC}
\label{tabla_pugh_adc_termistores}
\centering
\begin{tabularx}{\columnwidth}{|Y|Y|Y|Y|}
\hline
\rowcolor{darkblue} \textbf{Criterio} & \textbf{ADC Integrado (ESP32 / Arduino)} & \textbf{ADS1115} & \textbf{MCP3008 / MCP3208} \\
\hline
\rowcolor{white} Resolución & 0 & 2 & 1 \\
\hline
\rowcolor{lightblue} Precisión / Linealidad & 0 & 2 & 1 \\
\hline
\rowcolor{white} Velocidad de muestreo & 2 & 1 & 2 \\
\hline
\rowcolor{lightblue} Consumo energético & 2 & 1 & 1 \\
\hline
\rowcolor{white} Costo & 2 & 1 & 1 \\
\hline
\rowcolor{lightblue} Facilidad de integración & 2 & 2 & 1 \\
\hline
\rowcolor{white} Resistencia a la intemperie & 1 & 1 & 1 \\
\hline
\rowcolor{darkblue} Total & 9 & 11 & 7 \\
\hline
\end{tabularx}
\end{table}

\paragraph{Justificación de la selección}

Para la medición de voltajes de divisores resistivos de termistores, los ADC externos de alta resolución como el ADS1115 permiten obtener mayor precisión y estabilidad frente a fluctuaciones del microcontrolador, sin requerir una solución sobredimensionada. Su arquitectura sigma-delta y resolución de 16 bits lo hacen adecuado para mediciones de temperatura en tiempo real, manteniendo bajo costo y facilidad de integración en sistemas distribuidos. \\

\subsubsection{Filtro de ruido}

El acondicionamiento de señal en sistemas embebidos requiere minimizar el ruido proveniente de fuentes internas o externas al sistema. Para ello, se evaluaron diferentes estrategias de mitigación y filtrado considerando su efectividad, costo, facilidad de implementación y compatibilidad con el hardware de adquisición.

\begin{table}[H]
\caption{Evaluación de Técnicas de Mitigación de Ruido}
\label{tabla_pugh_ruido}
\centering
\begin{tabularx}{\columnwidth}{|c|Y|Y|Y|Y|}
\hline
\rowcolor{darkblue} \textbf{Criterio} & \textbf{Capacitores de desacople} & \textbf{Filtros RC pasivos} & \textbf{Filtros activos (OpAmp)} & \textbf{Blindaje electromagnético} \\
\hline
\rowcolor{white} Efectividad frente a ruido de alta frecuencia & 2 & 1 & 2 & 1 \\
\hline
\rowcolor{lightblue} Costo & 2 & 1 & 0 & -1 \\
\hline
\rowcolor{white} Complejidad de implementación & 2 & 1 & -1 & -1 \\
\hline
\rowcolor{lightblue} Impacto en el consumo energético & 2 & 1 & 0 & 1 \\
\hline
\rowcolor{white} Compatibilidad con sistemas embebidos & 2 & 1 & 0 & 1 \\
\hline
\rowcolor{darkblue} Total & 10 & 5 & 1 & 1 \\
\hline
\end{tabularx}
\end{table}

\paragraph{Justificación de la Selección}

Los capacitores de desacople se seleccionan como la opción más adecuada para el acondicionamiento de señal en el sistema propuesto. Su bajo costo, facilidad de integración y alta efectividad en la atenuación de interferencias de alta frecuencia los convierten en la estrategia óptima para estabilizar las líneas de alimentación y reducir el ruido inducido por los transitorios de conmutación. Además, su aplicación localizada cerca de los pines de alimentación de los componentes críticos contribuye significativamente a la estabilidad general del sistema. \\

\subsubsection{Microcontrolador}

La elección del microcontrolador define las capacidades de procesamiento, comunicación y gestión energética de los sistemas de monitoreo solar. Entre los criterios relevantes para la selección se incluyen la integración de comunicaciones inalámbricas, el consumo energético, el costo, la facilidad de desarrollo y la disponibilidad comercial. La evaluación de diferentes alternativas permite identificar plataformas que balanceen adecuadamente estas características.

\begin{table}[H]
\caption{Evaluación de microcontroladores}
\label{tabla_pugh_microcontroladores}
\centering
\begin{tabularx}{\columnwidth}{|Y|Y|Y|Y|Y|}
\hline
\rowcolor{darkblue} \textbf{Criterio} & \textbf{ESP32-S3} & \textbf{TTGO ESP32 LoRa32 V2.1} & \textbf{Arduino MKR WAN 1310} & \textbf{STM32L4 + LoRa Module} \\
\hline
\rowcolor{white} Comunicación integrada & 1 & 2 & 0 & -1 \\
\hline
\rowcolor{lightblue} Consumo energético & 0 & 1 & 1 & 2 \\
\hline
\rowcolor{white} Costo & 0 & 1 & -1 & -1 \\
\hline
\rowcolor{lightblue} Facilidad de desarrollo & 1 & 1 & 0 & -1 \\
\hline
\rowcolor{white} Disponibilidad & 1 & 0 & -1 & 1 \\
\hline
\rowcolor{lightblue} Capacidad de procesamiento & 2 & 1 & 0 & 1 \\
\hline
\rowcolor{darkblue} Total & 5 & 6 & -1 & 1 \\
\hline
\end{tabularx}
\end{table}

\paragraph{Justificación de la selección}

La evaluación sugiere que las plataformas basadas en ESP32 ofrecen un balance adecuado entre procesamiento, consumo energético y comunicación integrada, lo que las hace apropiadas para sistemas distribuidos de medición solar. En particular, las variantes que incluyen soporte para LoRa permiten la transmisión inalámbrica de datos sin necesidad de seleccionar y compatibilizar módulos de comunicación por separado, lo que simplifica el diseño del sistema y reduce la complejidad de integración. \\

\subsubsection{Plataforma de análisis para IoT}

La gestión de los datos adquiridos requiere un sistema confiable que permita almacenamiento histórico, visualización y acceso remoto. Se compararon varias plataformas considerando criterios de facilidad de integración, capacidades de visualización, costo, flexibilidad y disponibilidad de documentación.

\begin{table}[H]
\caption{Evaluación de plataformas de análisis para IoT}
\label{tabla_pugh_cloud}
\centering
\begin{tabularx}{\columnwidth}{|c|Y|Y|Y|Y|}
\hline
\rowcolor{darkblue} \textbf{Criterio} & \textbf{ThingSpeak (MathWorks)} & \textbf{AppScripts + Google Drive} & \textbf{Grafana + InfluxDB} & \textbf{AWS IoT / DynamoDB} \\
\hline
\rowcolor{white} Facilidad de integración & 2 & 1 & 0 & -1 \\
\hline
\rowcolor{lightblue} Capacidades de visualización & 2 & 0 & 2 & 1 \\
\hline
\rowcolor{white} Costo & 1 & 2 & 1 & 0 \\
\hline
\rowcolor{lightblue} Flexibilidad / Escalabilidad & 1 & 0 & 2 & 2 \\
\hline
\rowcolor{white} Documentación / Soporte & 2 & 1 & 1 & 1 \\
\hline
\rowcolor{darkblue} Total & 8 & 4 & 7 & 3 \\
\hline
\end{tabularx}
\end{table}

\paragraph{Justificación de la selección}

ThingSpeak se selecciona como la plataforma más adecuada debido a su fácil integración con microcontroladores y sistemas IoT, soporte nativo para visualización en tiempo real, y compatibilidad con MATLAB para procesamiento adicional de datos. Esta solución permite almacenar, procesar y graficar los datos sin necesidad de configurar complejas arquitecturas de base de datos, lo que simplifica el diseño general del sistema y acelera el desarrollo. \\

\subsection{Síntesis de la Arquitectura del Sistema}

La selección de componentes realizada en las secciones anteriores permite consolidar una arquitectura integral para el sistema de monitoreo solarimétrico, orientada a la adquisición precisa y transmisión confiable de variables ambientales y eléctricas. El diseño se fundamenta en una estrategia de procesamiento distribuido tipo \textit{Edge-Cloud}, combinando adquisición embebida local con análisis remoto.

En la capa \textbf{Edge}, el módulo TTGO LoRa32 V2.1, basado en el microcontrolador ESP32, ejecuta las tareas de adquisición y preprocesamiento de datos. Los elementos seleccionados \textit{—célula solar de referencia KXOB25-14X1FTR-ND, INA226 para medición de $I_{SC}$, ADS1115 para la lectura de termistores USP10982—} proporcionan entradas digitales al microprocesador por protocolo I2C de alta resolución (16 bits). La etapa de acondicionamiento incluye capacitores de desacople distribuidos para asegurar la estabilidad y minimizar interferencias electromagnéticas a nivel analógico y media móvil para procesar los datos a nivel digital. Además, se mantiene respaldo de datos de manera local mediante almacenamiento SD.

En la capa \textbf{Gateway/Transmisión}, la conectividad LoRa integrada en el TTGO elimina la necesidad de módulos externos, simplificando la topología de comunicación y reduciendo el consumo energético. Los datos son enviados a intervalos definidos hacia la nube mediante una pasarela \textit{TTGO - Gateway}.

Finalmente, la capa \textbf{Cloud} emplea \textit{ThingSpeak} de MathWorks como plataforma de gestión, almacenamiento y visualización de datos. Esta solución ofrece compatibilidad nativa con MATLAB, facilitando la aplicación de modelos analíticos, algoritmos de calibración y generación de indicadores de desempeño.

En conjunto, la arquitectura seleccionada logra un equilibrio entre costo, precisión, eficiencia energética y escalabilidad, garantizando una base sólida para el desarrollo de un sistema solarimétrico de monitoreo modular y científicamente reproducible.

% -------------------------------------------------------------------
% CAPÍTULO 5: DESCRIPCIÓN DETALLADA DE LA SOLUCIÓN
% -------------------------------------------------------------------

\section{Descripción Detallada de la Solución}

\subsection{Generalidades del sistema}

El sistema desarrollado es un instrumento solarimétrico distribuido para medir de forma confiable la irradiancia global en plano horizontal (GHI) utilizando células fotovoltaicas como transductores primarios con compensación térmica y calibración estadística frente a un sensor comercial de referencia. La solución adopta una arquitectura híbrida \textit{Edge--Cloud} y opera como nodo IoT dentro de la red del SESlab, priorizando:

\begin{enumerate}
    \item Bajo costo.
    \item Replicabilidad.
    \item Precisión metrológica y trazabilidad.
    \item Robustez y operación continua en exteriores.
    \item Registro redundante y telemetría de bajo consumo.
\end{enumerate}

El proyecto se planteó como un caso de estudio que, además de validar la técnica de estimación de irradiancia basada en la corriente de cortocircuito fotovoltaica, evalúa los componentes y las técnicas de acondicionamiento de señal que posibilitan el sensor. Con ello se establecieron precedentes para que futuros proyectos puedan invertir recursos de investigación (dinero, tiempo y mano de obra) en optimizar, miniaturizar y orientar el diseño hacia producción en masa con suficientes garantías de éxito. Bajo esta idea, el proyecto se desarrolló sobre dos enfoques simultáneos y  complementarios. \\

\subsubsection{Enfoque teórico}

Se evaluó, analizó y registró información proveniente de seis \textit{líneas} de adquisición paralelas para la estimación de irradiancia solar. En las etapas iniciales del flujo de información (\textit{célula, INA226, termistor, divisor de tensión y capacitores de desacople}) se implementaron seis canales equivalentes, lo que permitió recopilar suficientes datos para una evaluación estadística robusta tanto de la técnica como de los componentes y fabricantes seleccionados.

Para este fin, en la tarjeta SD se almacenaron, precedidos por marcas de tiempo, los valores de $I_{SC}$, $T$ y $G$ de cada \textit{línea/sensor} individual, así como el promedio de las seis líneas, con un tiempo de muestreo de 15\,s. En la base de datos de ThingSpeak se reporta $G$ de las seis células y el promedio.

El estudio exhaustivo de validación de componentes queda fuera del alcance del presente trabajo, ya que requiere un período extenso de adquisición (6--12 meses). Del mismo modo, los análisis estadísticos necesarios para determinar la cantidad mínima de \textit{líneas} paralelas que garanticen integridad y precisión de las mediciones también exceden el alcance de este proyecto. No obstante, se incluyó un estudio preliminar sobre estos temas en la Sección VI (Análisis de Resultados). Se insta al lector a interpretar dichos resultados con cautela, pues se basan en datos de una semana post-calibración, potencialmente insuficientes para conclusiones definitivas.

En consecuencia, los procesos aquí implementados priorizaron la recopilación, robustez y trazabilidad metrológica de los datos, de modo que, conforme pase el tiempo, se genere una base de datos sólida y completa para que investigaciones futuras aborden los estudios que este proyecto deja abiertos. \\

\subsubsection{Enfoque práctico}

En paralelo, el SESlab requiere capacidad funcional inmediata como nodo de medición de irradiancia. Para lograrlo, se asumió que eventuales diferencias estadísticas entre los componentes de adquisición primarios (\textit{célula, INA226, termistor, divisor de tensión y capacitores de desacople}) pueden compensarse mediante el \textit{promedio entre líneas}. Así, para fines operativos, los resultados reportados corresponden al promedio del sistema.

En la práctica, se publica $G$ en ThingSpeak para cada una de las seis células y para el promedio del sistema, aplicando un promedio temporal de 5\,min para atenuar la variabilidad rápida propia de nubosidad y efectos atmosféricos. En el \textit{Gateway} del SESlab se reporta únicamente el promedio de las seis células, también como promedio temporal de 5\,min.

Con estas medidas, se completaron pruebas, configuraciones y calibraciones para operar como nodo activo dentro de la red distribuida de sensado solarimétrico del SESlab. En síntesis, las seis células funcionan como una sola a través de dos promedios complementarios: (i) \textit{espacial} (entre líneas) y (ii) \textit{temporal} (ventana de 5\,min). \\


\subsubsection{Arquitectura general del sistema}
La arquitectura se organiza en tres capas acopladas por interfaces bien definidas:

\paragraph{Capa Edge (Nodo de medición)} Se encarga de la adquisición y acondicionamiento de datos, registro en memoria local y preparación de los paquetes de transmisión; la componen los siguientes elementos:
\begin{itemize}
    \item Transductor FV (célula de Si KXOB25-14X1F-TR operada en cortocircuito).
    \item Sensado de $I_{SC}$ mediante monitor de corriente de precisión INA226 de $16 bits$.
    \item Medición de temperatura de la célula con NTC USP10982 + ADC ADS1115 $\Delta\Sigma$ de $16 bits$.
    \item Técnicas de compensación de ruido mediante media movil (digital), capacitor 'Bulk' para mantener estable la tensión de línea de \SI{47}{\micro\farad}(analógico), capacitores de filtro pasa bajas RC para los divisores de tensión de la línea de medición de temperatura de \SI{0.1}{\micro\farad} (analógico) y capacitores de desacople para las alimentaciones de sensores de \SI{0.1}{\micro\farad} (analógico).
    \item Cómputo embebido (TTGO ESP32 LoRa V2) para adquisición, compensación térmica, estimación de $G$, almacenamiento local, preparación de paquetes HTTP para Thinkspeak y paquetes tipo 'buffer' para transmisión LoRa.
    \item Pantalla OLED para mantenimiento y tareas de corrección y calibración en campo.
\end{itemize} 

\paragraph{Capa Gateway} Recibe tramas LoRa desde múltiples nodos, valida integridad (CRC, longitud, campos), sella tiempo en recepción, aplica reglas básicas de \textit{ingest} (descarte de duplicados, colas de reintento, etc.), y reenvía a la base de datos en Firebase. El gateway mantiene sincronía NTP y métricas de radio (RSSI/SNR) para diagnóstico de enlace.

\paragraph{Capa Cloud} Provee persistencia histórica, visualización y servicios API. Se almacenan canales con capacidad para siete u ocho fields (seis células, promedio y field de la calibración). La nube hospeda además cuadernos de análisis, gráficos de desempeño y utilidades de exportación, gráficas de visualización y correlación, y tiene potencial de análisis directo con Matlab para crear nuevos gráficos y análisis según necesidad.

\begin{figure}[H]
    \centering
    \includegraphics[width=0.8\textwidth]{Images/Diagramas/Diagrama_de_Bloques.png}
    \caption{Diagrama de bloques general del sistema propuesto. Fuente: Elaboración propia}
    \label{fig:System_Block_Diagram_2}
\end{figure}

\subsubsection{Repositorios}
La solución se documentó con repositorios y artefactos versionados para asegurar reproducibilidad:

\paragraph{Software del nodo} Todo el código fuente \textit{-configuración de buses (I\textsuperscript{2}C, SPI), drivers de sensores, tareas periódicas, colas de adquisición, compensación y transformación de datos, empaquetado LoRa, servicios de mantenimiento (OLED/menús, Wi-Fi/NTP), manejo de SD-}  y demás archivos relacionados con la programación del sistema se pueden acceder desde el repositorio de Github ``SolarIrradianceNode'', URL: \url{https://github.com/Ing-Adrian-RM/SolarIrradianceNode.git}

\paragraph{Bases de datos (Cloud IoT, Gateway)} Las bases de datos IoT de Thinkspeak se pueden acceder por tres canales distintos, cada uno contiene siete 'fields', las irradiancias de las seis células y el promedio, a excepción del canal de calibración que contiene en el octavo field los valores de irradiancia del sensor de referencia Spektron 210 utilizado durante el proceso de calibración.

\begin{itemize}
    \item \textit{Calibración:} Se puede acceder a los datos utilizados para la calibración del sistema en el canal público de Thinkspeak ´´Calibration -> 1 min Sample Time'', URL: \url{https://thingspeak.mathworks.com/channels/3108384}
    \item \textit{Base datos $15 s$:} El canal que almacena los datos de muestreo de $15 s$ ´´Solar Irradiance -> 15 sec Sample Time'' se puede acceder mediante el URL: \url{https://thingspeak.mathworks.com/channels/3116915}
    \item \textit{Base de datos $5 min$:} El canal que registra los datos promediados temporalmente cada $5 min$ ´´Solar Irradiance -> 5 min Sample Time`` es accesible mediante el URL: \url{https://thingspeak.mathworks.com/channels/3100981}
\end{itemize}

Los tres canales son accesibles en vista pública mediante el URL adjunto, sin embargo, este tipo de acceso no permite control total sobre las bases de datos. En caso de necesitarse dicho control para generar nuevas API's vinculadas a las bases de datos, gráficos o análisis de Matlab, exportación de los datos, entre otros, se debe referir la solicitud de los accesos al equipo de investigación del SESlab.

Por otra parte, en el caso del Gateway, los datos recopilados son cargados a una base de datos de Firebase y desplegados en la página web ``Cosecha-BioCarbón'' URL: \url{https://cosecha-6aadb.web.app}

\paragraph{Documentación técnica y CAD/PCB} Los archivos de documentación técnica, informes de calibración, caracterización, materiales y costos, esquemáticos, layouts, referencias, datasheet, imágenes y demás archivos referentes al proyecto se pueden acceder en el repositorio de Google Drive ``Sensor de Irradiancia'' URL: \url{https://drive.google.com/drive/folders/1yivz0Nfw8G2uwzFmgZQO70k2WL2UrMPG?usp=sharing} \\

\subsubsection{Descripción de la carcasa}
     % Caja de plexo IP65, con modificaciones de ventilación y malla "caseras", conectores de manguito para entradas de cables, escuadras metálicas como sistema de sujeción

 \begin{figure}[h]
    \centering
    \begin{subfigure}{0.48\linewidth}
    % Reemplazar por foto del montaje/placa
    \fbox{\rule[0pt]{0pt}{0.28\textwidth}\rule[0.48\linewidth]{0pt}{0pt}\parbox[c][0.28\textwidth][c]{\linewidth}{\centering \textit{Placeholder: Foto de la placa TTGO + cableado.}}}
    \caption{Vista del ensamblaje.}
    \label{fig:hw-foto}
    \end{subfigure}\hfill
    \begin{subfigure}{0.48\linewidth}
    % Reemplazar por esquema del divisor NTC
    \fbox{\rule[0pt]{0pt}{0.28\textwidth}\rule[0.48\linewidth]{0pt}{0pt}\parbox[c][0.28\textwidth][c]{\linewidth}{\centering \textit{Placeholder: Esquema del divisor NTC y RC.}}}
    \caption{Esquema del divisor NTC + RC.}
    \label{fig:divisor-ntc}
    \end{subfigure}
    \caption{Plantillas de imagen para documentación del hardware.}
\end{figure}

\subsubsection{Esquemático}
     
\subsubsection{Placa de Circuito Impreso (PCB)}
    % Medidas, técnicas de soldado, elementos, fotos, ancho de pistas, tamaño de pads y vías

\subsubsection{Sincronización temporal y sellado de tiempo}
El sello de tiempo se gestiona en UTC para evitar ambigüedades por horario local. Los protocolos establecen tres tipos de sellado de tiempo:
\begin{itemize}
    \item \textit{Sellado para almacenamiento local:} Se utiliza la conexión WiFi del TTGO para sincronizar con microcontrolador con los servidores NTP con configuración para sincronizar el tiempo UTC-6 (hora para Costa Rica) y sellar las mediciones de respaldo local en SD.
    \item \textit{Sellado para nube IoT Thinkspeak:} La plataforma de IoT de Mathworks, automatiza el sellado de tiempo de cada carga de datos cuando llegan al servidor.
    \item \textit{Sellado para Gateway:} En el caso de la transmisión hacia el Gateway, este se encarga de realizar el sellado cuando los datos llegan a la Raspberry Pi 4, de manera que se limitan los errores de sincronización entre los distintos nodos distribuidos.
\end{itemize}

Con estas configuraciones, todas bajo sincronización por UTC, se garantiza la trazabilidad temporal de las mediciones. Además, se incorporaron protocolos de verificación de sellos de tiempo para que los muestreos que no cumplan con condiciones de sellado válidas sean descartados; de esta manera, no se introducen errores temporales en las bases de datos. \\

\subsubsection{Lista de materiales (BOM)}

\subsection{Descripción del hardware}

Esta sección detalla la arquitectura física del sistema, los módulos empleados, las conexiones eléctricas relevantes y las consideraciones de diseño de PCB/ensamble. 

\subsubsection{Módulo de procesamiento}

La placa TTGO LoRa32 V2 es un versátil módulo de desarrollo que fusiona la potencia del microcontrolador ESP32 con capacidades de comunicación inalámbrica de largo alcance LoRa. Al integrar tanto Wi-Fi/Bluetooth a través del ESP32 como un transceptor LoRa (SX1276 con capacidad de frecuencia 868, 915 o 923 MHz), esta placa es ideal para proyectos de Internet de las Cosas (IoT) que requieren una combinación de conectividad local y de largo alcance. Además de un conector SMA para antena LoRa de $2 dBi$ externa que mejora la calidad y el alcance de la señal, la placa incluye un regulador de voltaje a \SI{3.3}{\volt} y salida de $5 V$ para alimentar sus componentes y periféricos. Proporciona buses de comunicación I²C y SPI, permitiendo fácil conexión de diversos sensores, actuadores y periféricos adicionales para expandir sus funcionalidades. También incorpora una pantalla OLED integrada de 0.96 pulgadas para mostrar información en tiempo real, un conector para batería Li-Po (no se utiliza actualmente pero proporciona potencialmente capacidad de funcionamiento autónomo) y una ranura para tarjetas MicroSD, ofreciendo una solución completa para proyectos portátiles y de registro de datos.

\begin{figure}[H]
    \centering
    \includegraphics[width=0.5\textwidth]{Images/Componentes/TTGO.png}
    \caption{TTGO LoRa32 V2.1\_1.6 de la marca Lilygo \cite{ttgo_buy}.}
    \label{fig:ttgo}
\end{figure}

\subsubsection{Módulo de transducción:}
El módulo de transducción es el encargado de convertir las variables físicas del entorno (radiación solar y temperatura) en señales eléctricas medibles y procesables por la electrónica del sistema. Para ello, se emplean dos componentes principales: una célula solar para la medición fotovoltaica y un térmistor para la medición térmica.

\paragraph{Transducción fotovoltaica $\rightarrow$ corriente}
Para la medición de la intensidad de la radiación solar incidente se utiliza la célula solar IXOLARTM SolarBIT de línea de productos de ANYSOLAR de silicio monocristalino modelo KXOB25-14X1F \cite{ixolartm_datasheet}. Al incidir la radiación solar sobre las células, estas generan una corriente de cortocircuito proporcional al valor de irradiancia, y mediante transformaciones digitales basadas en los datos de la hoja de datos del fabricante se pueden encontrar coeficientes de proporcionalidad, compensación térmica y calibración para obtener el valor final de irradiancia. Las características físicas y de diseño de esta célula, como su pequeño tamaño y alta eficiencia $\eta \approx 25\%$, además de que pueden operar en condiciones de baja luminosidad debido a su respuesta espectral amplia, su bajo costo y gran disponibilidad comercial, la vuelven una excelente opción de transducción primaria para proyectos de sensórica que requieren miniaturizar las dimensiones. 

\begin{figure}[H]
    \centering
    \includegraphics[width=0.55\textwidth]{Images/Componentes/ANYSOLAR.png}
    \caption{IXOLARTM SolarBIT de ANYSOLAR modelo KXOB25-14X1F \cite{ixolartm_buy}.}
    \label{fig:solar_bit}
\end{figure}

Con su pequeño tamaño de apenas $(23\text{x}8\text{x}1.8) mm$ estas células generan de manera estable en condiciones SC (\textit{Standard Condition: $1~sun = 1000 W/m^2; Air~Mass = 1.5; T = 25 ^\circ C$}) una corriente de cortocircuito $I_{SC} = 58.6 mA$, lo cual representa una magnitud dentro del margen de trabajo de instrumentos de medición de precisión comerciales y de bajo costo como el INA226. Además, se puede observar en la Figura \ref{fig:curva_I-V_IXOLAR} como la curva de corriente - voltaje (color azul), presenta un comportamiento ``plano'' para voltajes de trabajo entre $[0 - 0.45] V$ aproximadamente, lo cual genera la posibilidad de utilizar técnicas de medición de corriente tipo resistencia de shunt que generan un corrimiento del punto de trabajo desde $I_{SC}$ hacia $V_{OC}$ sin detrimento de la señal de corriente esperada, manteniendo la proporcionalidad de la señal recibida con la irradiancia.

\begin{figure}[H]
    \centering
    \includegraphics[width=0.7\textwidth]{Images/Gráficos/Curva_I-V_IXOLAR.png}
    \caption{Curva I-V de las células KXOB25-14X1F en SC digitalizada a con herramientas de IA a partir de la hoja de datos Fuente: Elaboración propia.}
    \label{fig:curva_I-V_IXOLAR}
\end{figure}

Una de las características más importantes de estas células solares es su respuesta espectral amplia que permite la captación de irradiancia solar en un rango de $[300 - 1200] nm$ aproximadamente, con una respuesta ``plana'' de casi el $100 \%$ en la mayoría del espectro de trabajo. Esto permite una captación de aproximadamente el $[85 - 90] \%$ del espectro energético objetivo.

\begin{figure}[H]
    \centering
    \includegraphics[width=0.7\textwidth]{Images/Gráficos/Respuesta_espectral.png}
    \caption{Respuesta espectral de las células KXOB25-14X1F \cite{ixolartm_datasheet}.}
    \label{fig:espectral_response}
\end{figure}

\paragraph{Transducción térmica $\rightarrow$ voltaje}
La medición de la temperatura ambiental o superficial se realiza mediante un termistor NTC (Negative Temperature Coefficient) de $10k\Omega$ modelo USP10982, de la serie USP de la marca Littlefuse. La resistencia eléctrica de este componente disminuye a medida que aumenta la temperatura. Para convertir esta variación de resistencia en una señal de voltaje linealizada y medible, el termistor se integró en un circuito divisor de voltaje modificado para funcionar como filtro RC. El voltaje resultante de este divisor de voltaje es directamente proporcional a la temperatura, y esta señal es la que se acondiciona y digitaliza posteriormente. Las características que situan este termistor una buena opción para el proyecto son principalmente su facilidad de integración, la cubierta plástica que recubre tanto el NTC como el cableado y le da una buena capacidad para trabajar en exteriores y su coeficiente $Beta (K) = 3892$  \cite{ups10982_datasheet} bien definido dentro de un rango de temperaturas de $[0-50 ^\circ C]$ que se acopla bien a las temperaturas esperadas en la puesta en marcha del proyecto. Para la conversión de los valores de voltaje a temperatura se utiliza la fórmula ``Beta Simplificada'' que, como su nombre lo indica, es una simplificación de la ecuación de Steinhart-Hart. Los cálculos matemáticos de esta transformación se abordarán más adelante en el módulo de sensores de la descripción del software.

\begin{figure}[H]
    \centering
    \includegraphics[width=0.5\textwidth]{Images/Componentes/UPS10982.png}
    \caption{Termistor de coeficiente negativo (NTC) UPS10982 \cite{ups10982_buy}.}
    \label{fig:UPS10982}
\end{figure}

\subsubsection{Módulo de sensado de corriente}
El sensado de corriente se implementa con el monitor de potencia INA226 de Texas Instruments \cite{ina226_datasheet}. Este CI integra un ADC de \SI{16}{\bit} para medir la caída de tensión en una resistencia shunt, calcular la corriente y estimar la potencia del canal(tiene la capacidad pero no se utilizó). Se comunica mediante bus I\textsuperscript{2}C, lo que facilita su integración con el ESP32 de la TTGO. Entre sus características relevantes para este proyecto se encuentran:
\begin{itemize}
    \item Medición diferencial de tensión en shunt con LSB (resolución) de \SI{2.5}{\micro\volt}.
    \item Rango típico de voltaje en shunt de hasta \(\pm\)\SI{81.92}{\milli\volt}.
    \item Tiempos de conversión y promediado programables (promedio interno de múltiples muestras).
    \item Direccionamiento I\textsuperscript{2}C mediante pines de dirección (hasta 16 direcciones I\textsuperscript{2}C).
\end{itemize}

\begin{figure}[H]
    \centering
    \includegraphics[width=0.42\textwidth]{Images/Componentes/INA226.png}
    \caption{Módulo/breakout INA226 utilizado para sensado de $I-{SC}$ \cite{ina226_buy}.}
    \label{fig:ina226}
\end{figure}

\paragraph{Topología de medición (shunt + entrada diferencial) y selección de la resistencia shunt}
Para medir la corriente generada por la célula se usa una resistencia shunt de bajo ohmiaje en serie con el cátodo de la célula; el INA226 mide diferencialmente la caída $V_{sh}$ y, con la calibración apropiada, reporta corriente. Se adoptan buenas prácticas de selección como shunt de precisión (tolerancia $\leq 1\%$ , bajo $TCR = \pm 50 (ppm/^\circ C)$), y desacoplo local de \SI{0.1}{\micro\farad}.

Para evaluar la viabilidad de la resistencia de shunt en el CI se deben considerar los criterios globales de diseño del proyecto, de manera que para realizar el análisis de rendimiento del $R_{sh}$ se siguieron las siguientes prioridades:

\begin{itemize}
    \item Se debe garantizar la seguridad del sistema, es decir $V_{sh} = I_{SC} * R_{sh} < 81.92 mV$.
    \item Se debe considerar la desviación del punto de trabajo de cortocircuito que aumenta conforme aumenta el tamaño del resistor y su impacto en la proporcionalidad de la salida con la irradiancia.
    \item Se debe mejorar la resolución de corriente $LSB_I$ cuanto sea posible, y dicha resolución mejora conforme aumenta $R_{sh}$ y decrece $LSB_I = LSB_V/R_{sh}$.
\end{itemize}

Basado en los criterios anteriores, se debe alcanzar un compromiso de seguridad y precisión del sistema. Primero se consideró el resistor de shunt original del CI $R_{sh} = 2m\Omega$ de la siguiente manera: 
 
 La corriente máxima esperada es la corriente de cortocircuito:

\[I_{SC,\max} = \SI{58.6}{\milli\ampere}\]

El voltaje en shunt viene dado por:

\begin{equation}
    V_{sh} = I R_{sh}    
\end{equation}

Donde 
\[V_{sh}(max) = \pm81.92 mV\]

Evaluando el resistor original se tiene: $R_{sh,orig}= 2 m\Omega$
\begin{gather}
    V_{sh,orig} = 58,6\times10^{-3} \times 2\times10^{-3} = 117.2 \mu V \\
    LSB_{I,orig} = \frac{2.5\times 10^{-6}}{2\times10^{-3}} = 1.25 mA \\
\end{gather}

Dado que el $V_{sh,orig} = 117.2 \mu V$ existe un gran margen de mejora, aunado a que un $LSB_{I,orig} = 1.25 mA$ es demasiado grande y reduce mucho la capacidad de resolución del sistema ante cambios pequeños de irradiancia. Por otra parte, $V_{sh,orig}$ representa el punto de trabajo que se mantiene muy cerca del cortocircuito. 

Se evaluó el resistor RNCL2512FTR500 y se obtuvo para $R_{sh,sel}= 0.5 \Omega$ \cite{resistor.5_buy}:
\begin{gather}
    V_{sh,sel} = 58,6\times10^{-3} \times 0.5 = 29.3 mV \\
    LSB_{I,sel} = \frac{2.5\times 10^{-6}}{0.5} = 5 \mu A
\end{gather}

En este caso $V_{sh,sel} = 29.3 mV$, se mantiene dentro del margen esperado sin acercarse a puntos de saturación cercanos a $V_{sh}(max)$, además $LSB_I = 5 \mu A$ lo cual brinda al sistema una capacidad de resolución $250$ veces mayor al resistor original. Se evaluó el corrimiento del punto de trabajo gráficamente como una recta ascendente que representa el cambio de voltaje con la resistencia con una pendiente $m = 1/R_{sh,sel}$ y su intersección con la curva I-V de la célula.

\begin{figure}[H]
    \centering
    \includegraphics[width=0.8\textwidth]{Images/Gráficos/Punto_de_trabajo_0.5.png}
    \caption{Punto de trabajo con $R_{sh,sel} = 0.5 \Omega$. Fuente: Elaboración propia.}
    \label{fig:WP}
\end{figure}

Se puede observar de la Figura \ref{fig:WP} que gracias a la estabilidad de la respuesta eléctrica de la célula seleccionada para el proyecto, el punto de trabajo está muy por debajo del $[0 - 0.45] V$ donde la curva I-V es casi plana, por lo que no implica detrimento alguno la utilización de la $R_{sh,sel} = 0.5 \Omega$ en la proporcionalidad de la corriente medida y la irradiancia.

\paragraph{Pérdida de potencia y disipación térmica} 

La disipación en $R_{sh,sel}$ con $I_{SC}$ es:

\[P_{sh} = I_{SC}^2 R_{sh,sel} = (58.6\times10^{-3})^2 \times 0.5 = 1.717 mW\]

negligible para un shunt SMD de $2W$, con margen térmico amplio y sin riesgo de auto-calentamiento significativo. 

\paragraph{Interfaz eléctrica}
El INA226 se alimenta a $5V$ y las líneas SDA/SCL cuentan con resistencias de pull-up. La referencia de tierra es única para toda la placa y se mantiene una topología de estrella entre la TTGO y los módulos de adquisición para minimizar la inyección de ruido. \\

\subsubsection{Módulo de sensado de temperatura}
La adquisición de temperatura se realiza con el conversor ADC ADS1115 de Texas Instruments \cite{ads1115_datasheet}. Es un ADC de \SI{16}{\bit}, interfaz I\textsuperscript{2}C, con cuatro entradas analógicas multiplexadas (dos diferenciales o cuatro de un solo extremo) y PGA interno. En este proyecto:
\begin{itemize}
    \item Se emplea en \textit{single-ended} (canales AIN\(_x\) respecto a GND) para leer el nodo del divisor NTC.
    \item Se aprovecha su PGA para adecuar el rango de entrada al voltaje del divisor, maximizando la resolución efectiva.
    \item Se utiliza su promediado interno para atenuar ruido de alta frecuencia, complementando el filtro RC analógico del nodo.
\end{itemize}

\paragraph{Interfaz con el NTC}
El termistor NTC USP10982 (\SI{10}{\kilo\ohm} @ \SI{25}{\celsius}) se conecta en divisor con una resistencia fija de \SI{10}{\kilo\ohm}. El punto medio alimenta un canal del ADS1115. El ADC se alimenta a \SI{5}{\volt} para una mayor resolución de modo común, asegurando que el rango de entrada siempre permanezca dentro de \([0, V_{DD}]\). El filtro RC (\(R_\text{th}\)–\(C\)) descrito en la sección \textit{\ref{sec:Control_ruido}. Módulo de control de ruido} fija la banda de paso y garantiza un \textit{settling time} compatible con el ritmo de muestreo.

\begin{figure}[H]
    \centering
    \includegraphics[width=0.38\textwidth]{Images/Componentes/ADS1115.png}
    \caption{Módulo/breakout ADS1115 de \SI{16}{\bit} para adquisición de temperatura \cite{ads1115_buy}.}
    \label{fig:ads1115}
\end{figure}

\subsubsection{Módulo de control de ruido}
\label{sec:Control_ruido}

El diseño de un sistema electrónico robusto, especialmente uno que incorpora múltiples módulos de comunicación inalámbrica y periféricos digitales (microcontrolador TTGO ESP32, Wi-Fi, LoRa, SD, pantalla OLED e interfaces SPI/I2C), requiere una cuidadosa estrategia de manejo de ruido. Por esta razón, se identificaron diversas fuentes de ruido como:

\begin{enumerate}
    \item \textit{Ruido de alta frecuencia:} Tecnologías como Wi-Fi y LoRa trabajan en rangos de frecuencias altas con portadoras de RF a \SI{915}{\mega\hertz} y \SI{2.4}{\giga\hertz} respectivamente, además, durante las ráfagas de transmisión y recepción de paquetes se generan demandas de corriente que generan picos y se manifiestan como pulsos de corriente de mayor duración en la línea de $5 V$, lo que ocasiona caídas de tensión que pueden afectar la precisión de los instrumentos; como es el caso del ADS1115, que es altamente dependiente de la tensión de alimentación para generar transformaciones A/D precisas.
    \item \textit{Actividad concurrente del ESP32:} Los accesos a la tarjeta SD, refrescamientos de la pantalla OLED, o las comunicaciones con los INA226 y ADS1115, pueden generar ruido de alta frecuencia en los canales SPI y I\textsuperscript{2}C debido a la conmutación del procesador del microcontrolador y los cambios rápidos en los flancos de reloj. 
\end{enumerate}

Para amortiguar los trenes de corriente de baja y media frecuencia identificados anteriormente, se requiere un capacitor de alta capacitancia que pueda almacenar suficiente energía y liberarla rápidamente, además de un conjunto de capacitores que puedan filtrar las interferencias de alta frecuencia. Este proyecto implementó las siguientes estrategias para abordar esta problemática:

\begin{enumerate}
    \item Desacople de carga en las líneas de alimentación de la placa PCB (bulk).
    \item Desacople local de alta frecuencia en la entrada de cada circuito integrado.
    \item Filtrado analógico en el nodo de medida de temperatura (RC pasabajas).
    \item Filtrado digital mediante promediado integrado en el ADS1115, promediado integrado en el INA226, promediado de líneas de adquisición y media móvil temporal.
\end{enumerate}

La Figura \ref{fig:filtros_ruido_analógico} identifica todos los capacitores utilizados en la PCB para minimizar los efectos indeseados del ruido. En rojo se observa el capacitor de Bulk del sistema, en azul los capacitores de los filtros RC de las líneas de adquisición de temperatura, y en amarillo los capacitores de desacople sobre las líneas de alimentación de los circuitos integrados de adquisición de datos (seis INA226 y dos ADS1115).

\begin{figure}[H]
    \centering
    \includegraphics[width=0.74\textwidth]{Images/Componentes/Cap_ruido.png}
    \caption{Placa PCB del sistema con los capacitores de control de ruido identificados. Fuente: Elaboración propia}
    \label{fig:filtros_ruido_analógico}
\end{figure}

\paragraph{Capacitor bulk}
Se instaló un capacitor de tantalio SMD  de \SI{47}{\micro\farad} en el punto de alimentación de las salidas de $5 V$ y $GND$ que alimentan la placa PCB, esto es justo al lado de las conexiones de los pines del TTGO. Su objetivo es compensar las caídas de tensión generadas por transmisiones y recepciones Wi-Fi y LoRa, y se utilizó tecnología de tantalio por su capacidad de almacenar suficiente energía y liberarla rápidamente para mantener las líneas de alimentación de la red de sensores estables.

\begin{figure}[H]
    \centering
    \includegraphics[width=0.4\textwidth]{Images/Componentes/Bulk.png}
    \caption{Capacitor TAJB476K010RNJ de la marca AVX de \SI{47}{\micro\farad} utilizado como Bulk \cite{bulk_buy}.}
    \label{fig:bulk}
\end{figure}

\paragraph{Filtros RC y capacitores de desacoplo.}
Para mitigar el ruido de alta frecuencia que podría acoplarse al sistema, se integraron capacitores cerámicos SMD. Estos componentes se eligieron por su baja ESR (Resistencia Equivalente en Serie), una característica crucial que minimiza el consumo de potencia y evita la reducción del voltaje de alimentación de los sensores.Los capacitores cumplen dos funciones principales en el diseño:

\begin{itemize}
    \item \textit{Desacoplo de Alimentación:} Los capacitores de desacoplo se ubican estratégicamente lo más cerca posible de los pines de \(5\,\text{V}\) y \(GND\) de cada circuito integrado, asegurando una fuente de energía limpia y estable.
    \item \textit{Filtrado en Nodos de Medición (RC Pasa-bajas):} En los puntos de medición de temperatura, los capacitores operan como filtros pasa-bajas RC.
\end{itemize}

En el caso del circuito de medición de temperatura, este se basa en un divisor de tensión compuesto por un termistor NTC USP10982 de $R_{NTC}(T) =10 k\Omega ~ @25 ^\circ C $ y una resistencia fija de $R_{fija} = 10 k\Omega$. El capacitor de filtrado se instala en paralelo con la NTC (conectado a tierra). Esta configuración transforma el nodo de lectura (la entrada del conversor ADS1115) en un filtro pasa-bajas de primer orden. Para un análisis detallado del rendimiento de este filtrado, se utiliza el equivalente de Thévenin del divisor de tensión. La resistencia en serie de Thévenin para este análisis es:

\begin{gather}
    R_{th} = R_{NTC}(T) \parallel R_{fija} \\
    R_{th} \approx 10 k\Omega \parallel 10 k\Omega \\
    R_{th} \approx 5 k\Omega
\end{gather}
    
\begin{figure}[H]
    \centering
    \includegraphics[width=0.8\textwidth]{Images/Diagramas/Cap_Thevenin.png}
    \caption{Equivalente de Thevenin visto por el ADS1115. Fuente: Elaboración propia}
    \label{fig:Thevenin}
\end{figure}

Con ello, la frecuencia de corte es:

\begin{equation}
    f_c = \frac{1}{2\pi R_{th} C},     
\end{equation}

Para determinar el tamaño adecuado del capacitor, se consideró inicialmente que las variaciones de temperatura son procesos lentos, de muy baja frecuencia. Idealmente, se podría optar por un capacitor de gran capacidad $ C \approx 2 \mu F$ para lograr una frecuencia de corte $f_c \approx 15 Hz$ que elimine eficazmente todo el contenido espectral no deseado.

Sin embargo, esta elección debe sopesarse con los requisitos globales del sistema. La adición de un filtro RC impone un tiempo de asentamiento (\textit{settling time}) en la línea de medición, lo cual es crítico al muestrear con el conversor ADS1115. Específicamente, al cambiar de canal en el multiplexor del ADC, se debe esperar un tiempo suficiente para que el filtro se estabilice antes de tomar la lectura.

Si bien la frecuencia de corte sugería un capacitor grande, esto implicaría un tiempo de asentamiento significativo, el tiempo de asentamiento se define como $t_{set}=5\tau$ (donde $\tau =R_{th}C$). Dado que el sistema utiliza tres canales por cada uno de los dos ADC disponibles, un capacitor grande generaría un consumo de tiempo total de $6\times t_{set}$ por cada ciclo de muestreo completo. Así las cosas, tras evaluar tanto las frecuencias de ruido que se desean filtrar como el impacto de los tiempos de asentamiento en el rendimiento del sistema, se optó por un valor de compromiso: capacitores de $C = 0.1 \mu F$. Esta elección resulta en un consumo de tiempo total por asentamiento de líneas en cada muestreo de:

\begin{equation}
    t_{respuesta} = 6t_{set} = 6 * 5\tau = 6 * 5 * 5 k * 0.1 \mu = 15 ms
\end{equation}

Esta configuración no compromete los tiempos de respuesta globales establecidos ($t_{respuesta} < 15 s$) por iteración de muestreo completo y permite filtrar ruido con una frecuencia de corte $f_c \approx 318 Hz$ que está muy por debajo de las frecuencias de ruido esperadas. Finalmente, se eligieron entre las opciones de mercado, los capacitores SMD cerámicos de \SI{0.1}{\micro\farad} GCM188R71C104KA37J de la marca Murata Electronics.

\begin{figure}[H]
    \centering
    \includegraphics[width=0.3\textwidth]{Images/Componentes/Cap_desac.png}
    \caption{Capacitor GCM188R71C104KA37J de la marca Murata Electronics de \SI{0.1}{\micro\farad} utilizado como filtros RC y desacopladores de alta frecuencia \cite{cap_ceram_buy}.}
    \label{fig:cap_desac}
\end{figure}

Por último, las técnicas de filtrado digital utilizadas se abordan más adelante en las secciones que corresponden para el INA226, el ADS1115 y el módulo de procesamiento digital. \\

\subsubsection{Módulo de comunicación}

El sistema integra dos canales de comunicación: LoRa de largo alcance (basado en el transceptor Semtech SX1276) y Wi-Fi/Bluetooth provistos por el ESP32 de la TTGO \cite{sx1276, ttgo_buy}. Esta combinación permite telemetría robusta a baja tasa de consumo con LoRa y conectividad web mediante Wi-Fi.

\paragraph{LoRa (SX1276)}
El SX1276 opera típicamente en las bandas de (\SI{868} - \SI{915} - \SI{923}{\mega\hertz}), con modulación LoRa (CSS) y anchos de banda configurables (\SI{7.8}{\kilo\hertz}–\SI{500}{\kilo\hertz}). Entre sus ventajas destacan:
\begin{itemize}
    \item Alta sensibilidad (del orden de $-130 dBm$), adecuada para enlaces de larga distancia a baja velocidad de datos.
    \item Factores de dispersión (\(SF\)) ajustables (SF7–SF12) que permiten intercambiar alcance por latencia/bit\_rate.
    \item Opción de antena externa vía conector SMA en la TTGO, mejorando el enlace RF y la direccionalidad.
\end{itemize}

\begin{figure}[H]
    \centering
    \includegraphics[width=0.46\textwidth]{Images/Componentes/SX1276.png}
    \caption{Transceptor LoRa SX1276 integrado en la TTGO para comunicación de largo alcance \cite{sx1276}.}
    \label{fig:sx1276}
\end{figure}

\paragraph{Wi-Fi / Bluetooth (ESP32)}
El ESP32 de la TTGO provee Wi-Fi 802.11 (b/g/n) en \SI{2.4}{\giga\hertz} y Bluetooth v4.2 (BR/EDR y BLE). Para este proyecto, el canal Wi-Fi se utiliza para:

\begin{enumerate}
    \item \textit{Servicio de enlace HTTP con Thinkspeak:} Subir lotes de datos cuando la red está disponible.
    \item \textit{Servicios locales:} Sincronización NTP, pruebas de conectividad y diagnósticos.
\end{enumerate}

El uso concurrente de Wi-Fi y operaciones de adquisición puede introducir ráfagas de consumo y ruido de alta frecuencia. Se mitiga con el \textit{bulk} de \SI{47}{\micro\farad}, desacoples locales y promediado (ver sección \ref{sec:Control_ruido}. Módulo de control de ruido), así como ventanas de adquisición sincronizadas con las tareas de comunicación.
    
\subsection{Descripción del software}
\subsubsection{Servicios}
    %Wifi -> UTC para timeStamps, Protocolos I2C & SPI, HTTP
\subsubsection{Módulo de sensores}
    % Calibración INA, configuración de registros, tiempos de muestreo, tipo de comunicación, etc
    % Calibración ADS1115, configuración de registros, tiempos de muestreo, tipo de comunicación, etc
    % Protocolos en código de ejecución de mediciones
\subsubsection{Módulo de procesamiento}
    % Transformación datos raw de los termistores por canales ADS a temp por formula Beta
    % Transformación Isc con compensación de temperatura a irradiancia
\subsubsection{Módulo de almacenamiento}
    % SD -> estructuras de datos csv Isc,Temp,Irra -> tiempos de muestreo -> promedios
\subsubsection{Módulo de visualización}
    % OLED, 4 tipos de pantallas de datos giratorias
    % Pantalla de estatus
    % Datos por panel
    % Datos promedio
    % Datos de transmisión RSSI, SNR
\subsubsection{Módulo de comunicación}
    % Nodo IoT LoRa, configuraciones, tiempos de transmisión, formato de buffer, promediados
    % Datos en la nube Thinkspeak, configuraciones, tiempos de transmisión, formato de url, promediados, graficos, API's

% -------------------------------------------------------------------
% CAPÍTULO 6: ANÁLISIS DE RESULTADOS
% -------------------------------------------------------------------

\section{Análisis de Resultados}

%Introducción a la sección

\subsection{Caracterización de celdas solares}
    % No se pudo con trazadores de curvas industriales porque son para paneles grandes y no alcanza la precisión
    % Se optó por trazar curvas I-V privando las células de luz (curva diodo) con máquinas Keysight y Venchbue -> análisis de precisión de instrumentos
    % Análisis de curvas I-V y modelo de celdas

En el caso del presente proyecto en particular, el modelado asumió que la corriente fotovoltaica $I_{\text{L}} = 0$ , debido a que no se tienen los medios para caracterizar las celdas con fuentes de luz controladas (como lámparas de ensayo). Esto permite estudiar la \textit{curva del diodo} en condiciones oscuras, identificando parámetros como $I_0$, $R_s$ y $R_{sh}$, que son fundamentales para la simulación y diseño del sistema de medición de irradiancia.

Este enfoque teórico permite analizar y predecir el comportamiento de las celdas solares incluso sin irradiancia directa, facilitando la calibración de sensores y el diseño de sistemas de medición confiables \cite{al-ezzi, bliss}.

\subsection{Evaluación del punto de trabajo}

\subsection{Calibración de unidades de adquisición y tiempos de muestro}
    
\subsection{Calibración del sistema por regresión lineal}
    \subsubsection{Métricas pre-calibración}
    \subsubsection{Métricas post-calibración teóricas}
    \subsubsection{Métricas post-calibración prácticas}
    \subsubsection{Deriva temporal y re–calibración}
    
\subsection{Modelo de error y presupuesto de incertidumbre}
La estimación de irradiancia se modela como:

\begin{equation}
\hat{G} \;=\; \alpha_{\!CAL}\,k_G\,\frac{I_{SC}(G,T)}{\,1+\alpha_T\,(T_{\text{cél}}-T_{\text{ref}})\,}\;+\;\beta_{\!CAL},
\label{eq:g_est}
\end{equation}

donde $\alpha_{\!CAL}$ y $\beta_{\!CAL}$ son los coeficientes de calibración (pendiente y offset), $k_G$ la sensibilidad efectiva, y $\alpha_T$ el coeficiente térmico relativo de la célula; $T_{\text{ref}}$ es típicamente 25\,\textdegree C. El presupuesto de incertidumbre se construye bajo el enfoque GUM por combinación en cuadratura de contribuyentes independientes:

\begin{equation}
u_c^2(\hat{G}) \approx
\left(\frac{\partial \hat{G}}{\partial I_{SC}}\right)^{\!2}\!u^2(I_{SC})
+\left(\frac{\partial \hat{G}}{\partial T_{\text{cél}}}\right)^{\!2}\!u^2(T_{\text{cél}})
+\left(\frac{\partial \hat{G}}{\partial \alpha_{\!CAL}}\right)^{\!2}\!u^2(\alpha_{\!CAL})
+\left(\frac{\partial \hat{G}}{\partial \beta_{\!CAL}}\right)^{\!2}\!u^2(\beta_{\!CAL})
+u^2_{\text{spec}}+u^2_{\text{ang}}+u^2_{\text{time}}+u^2_{\text{quant}}.
\end{equation}

Los términos adicionales recogen: \emph{mismatch} espectral ($u_{\text{spec}}$), error angular/coseno ($u_{\text{ang}}$), desalineación temporal nodo--referencia ($u_{\text{time}}$) y cuantización/noise floor de ADC ($u_{\text{quant}}$). La incertidumbre expandida se reporta como $U = k\,u_c(\hat{G})$ con $k\approx 2$ (95\,\% de cobertura). En explotación, se incluyen indicadores de calidad de dato por muestra (p.\,ej.,\ umbrales de $|$residuo$|$, saturación, temperatura fuera de rango, pérdida de sincronía) para trazabilidad y depuración.

En conjunto, estas generalidades establecen el marco de operación, trazabilidad y calidad sobre el que se describen, en secciones posteriores, los módulos de hardware y software, así como los procedimientos de calibración, validación y análisis de resultados.

    
\subsection{Análisis de datos}
    %Thinkspeak
\subsection{Respaldo de datos}
    %SD
\subsection{Desempeño del sistema}
\subsection{Desempeño del enlace LoRa}

% -------------------------------------------------------------------
% CAPÍTULO 7: CONCLUSIONES Y RECOMENDACIONES
% -------------------------------------------------------------------

\section{Conclusiones y Recomendaciones}
\subsection{Conclusión de objetivos}
\subsection{Limitaciones}
\subsection{Puntos de mejora}

% -------------------------------------------------------------------
% BIBLIOGRAFÍA
% -------------------------------------------------------------------

\begin{thebibliography}{99}

\bibitem{arrieta} % Artículo sobre potencial solar en Costa Rica
E. Arrieta, ``Costa Rica pierde oportunidad de explotar el tercer mejor potencial de energía solar del continente," La República, Sep. 23, 2024. [En línea]. Disponible:
\url{https://www.larepublica.net/noticia/costa-rica-pierde-oportunidad-de-explotar-el-tercer-mejor-potencial-de-energia-solar-del-continente}. Accessed: Oct. 12, 2025.

\bibitem{ice} % Informe del ICE sobre generación renovable
Instituto Costarricense de Electricidad, ``DOCSE\_2025\_Informe\_Generación\_Renovable\_2024," Dirección de Operación, Control y Gerencia del Sistema Eléctrico Nacional, 2024.

\bibitem{dunlop} % Guías industriales para medición de potencia fotovoltaica
E. D. Dunlop, F. Fabero, G. Friesen, W. Herrmann, J. Hohl-Ebinger, H. Mohring, H. Müllejans, N. Taylor, A. Virtuani, W. Warta, W. Zaaiman, and S. Zamini, ``Guidelines for PV Power Measurement in Industry."

\bibitem{king} % Método para mejorar la precisión de sensores de irradiancia de bajo costo usando fotodiodos y células fotovoltaicas
D.L. King, W.E. Boyson and B.R. Hansen, ``Improved accuracy for low-cost solar irradiance sensors,", Dec 31.

\bibitem{carrasco} % Sensado de irradiancia de bajo costo
M. Carrasco, A. Laudani, G. M. Lozito, F. Mancilla-David, F. Riganti Fulginei, and A. Salvini, ``Low-Cost Solar Irradiance Sensing for PV Systems," \emph{Energies}, vol. 10, no. 7, Jul. 2017.

\bibitem{cruz} % Diseño de medidor de irradiancia usando panel fotovoltaico
J. Cruz-Colon, L. Martinez-Mitjans, and E. I. Ortiz-Rivera, ``Design of a low cost irradiance meter using a photovoltaic panel," in \emph{Proc. 38th IEEE Photovoltaic Specialists Conference}, Jun. 2012.

\bibitem{orsetti} % Sistema confiable y económico para medición de irradiancia
C. Orsetti, M. Muttillo, F. R. Parente, L. Pantoli, V. Stornelli, and G. Ferri, ``Reliable and Inexpensive Solar Irradiance Measurement System Design," \emph{Procedia Engineering}, vol. 168, 2016, pp. 1767–1771.

\bibitem{rhiat} % Diseño y simulación de medidor solar de bajo costo
M. Rhiat, M. Karrouchi, A. Hassari, M. Melhaoui, I. Atmane, H. Azzaoui, M. El Ouariachi, J. Bouchnaif, P. Schmitz, and K. Hirech, ``Design and simulation of a low-cost solar irradiance meter for PV applications," \emph{E3S Web of Conferences}, vol. 469, 2023.

\bibitem{risdiyanto} % Registrador de radiación solar de bajo costo
A. Risdiyanto, A. A. Kristi, A. Junaedi, B. Susanto, N. A. Rachman, A. Muqorobin, H. P. Santosa, and A. Fudholi, ``Performance of low-cost solar radiation logger," \emph{International Journal of Electrical and Computer Engineering}, vol. 13, no. 4, Aug. 2023.

\bibitem{bliss} % Comparación interlaboratorios de mediciones FV
M. Bliss, T. Betts, R. Gottschalg, E. Salis, H. Müllejans, S. Winter, I. Kroeger, K. Bothe, D. Hinken, and J. Hohl-Ebinger, ``Interlaboratory comparison of short-circuit current versus irradiance linearity measurements of photovoltaic devices," \emph{Solar Energy}, vol. 182, Feb. 2019, pp. 256–265.

\bibitem{ibrahim} % Corriente de cortocircuito como herramienta de diagnóstico
A. Ibrahim, M. R. I. Ramadan, S. Aboul-Enein, A. Elsebaii, and S. M. El-Broullesy, ``Short Circuit Current $I_{sc}$ as a Real Non-Destructive Diagnostic Tool of Photovoltaic Modules Performance."

\bibitem{kopp} % Medición de irradiancia solar absoluta
G. Kopp, "Solar irradiance measurements."

\bibitem{dominguez} % Modelado de la irradiancia solar para optimización de producción agro
A. Domínguez-Álvarez, M. De-Tena-Rey and L. García-Moruno, '"Modelling global solar radiation to optimise agricultural production," Spanish journal of agricultural research : SJAR, vol. 19, no. 1, Jan 1, pp. e0201.

\bibitem{voyant} % Estudio sobre variabilidad estocástica de irradiancia solar
C. Voyant, A. Julien, M. Despotovic, G. Notton, L. A. Garcia-Gutierrez, C. F. Nicolosi, P. Blanc, and J. Bright, ``Stochastic coefficient of variation: Assessing the variability and forecastability of solar irradiance," \emph{Renewable Energy}, vol. 256, Jan. 2023, p. 123913.

\bibitem{wald} % Estudio general sobre radiación solar
L. Wald, '"BASICS IN SOLAR RADIATION AT EARTH SURFACE,".

\bibitem{marques} % Revisión de tecnología fotovoltaica
R. A. Marques Lameirinhas, J. P. N. Torres, and J. P. De Melo Cunha, ``A Photovoltaic Technology Review: History, Fundamentals and Applications," \emph{Energies}, vol. 15, no. 5, Mar. 2022.

\bibitem{al-ezzi} % Revisión de celdas solares fotovoltaicas
A. S. Al-Ezzi and M. N. M. Ansari, ``Photovoltaic Solar Cells: A Review," \emph{Applied Solar Innovations}, vol. 5, no. 4, Jul. 2021.

\bibitem{jho} % Comprensión del efecto fotoeléctrico
H. Jho, B. Lee, Y. Ji, and S. Ha, ``Discussion for the enhanced understanding of the photoelectric effect," \emph{European Journal of Physics}, vol. 44, no. 2, Feb. 2023.

\bibitem{dubey} % Efecto de la temperatura en la eficiencia fotovoltaica
S. Dubey, J. N. Sarvaiya, and B. Seshadri, ``Temperature Dependent Photovoltaic (PV) Efficiency and Its Effect on PV Production in the World – A Review," \emph{Energy Procedia}, vol. 33, pp. 311–321, 2013.

\bibitem{liu} % Análisis teórico de corriente de cortocircuito en sistemas FV
H. Liu, K. Xu, Z. Zhang, W. Liu, and J. Ao, ``Research on theoretical calculation methods of photovoltaic power short-circuit current and influencing factors of its fault characteristics," \emph{Energies}, vol. 12, no. 2, Jan. 2019.

\bibitem{khan} % Revisión de tecnologías de energía solar sostenible
J. Khan and M. H. Arsalan, ``Solar power technologies for sustainable electricity generation – A review," \emph{Renewable and Sustainable Energy Reviews}, vol. 55, Nov. 2016, pp. 414–425.

\bibitem{venkateswari} % Factores que influyen en la eficiencia de sistemas FV
R. Venkateswari and S. Sreejith, ``Factors influencing the efficiency of photovoltaic system," \emph{Renewable and Sustainable Energy Reviews}, vol. 101, Nov. 2019, pp. 376–389.

\bibitem{li} % Dependencia de la corriente de cortocircuito con la irradiancia
B. Li, A. Migan-Dubois, C. Delpha, and D. Diallo, ``Irradiance Dependence of the Short-Circuit Current Temperature Coefficient of sc-Si PV Module," in \emph{Proc. 47th IEEE Photovoltaic Specialists Conf. (PVSC)}, Jun. 2020, pp. 0564–0569.

\bibitem{tobon} % Estimación de parámetros de modelo fotovoltaico con optimización
A. F. Tobón Mejía, J. J. Rojas Montaño, S. I. Serna Garcés, and J. A. Herrera Cuartas, ``Estimación de los parámetros del modelo de un solo diodo del módulo fotovoltaico aplicando el método de optimización basado en búsqueda de patrones mejorado," \emph{Revista Ingeniería Universidad de Medellín}, vol. 20, no. 38, Apr. 2021, pp. 13–24.

\bibitem{garcía} % Modelado y validación de paneles fotovoltaicos
G. García Hombrados, ``Validación del modelado de paneles fotovoltaicos en diferentes condiciones de funcionamiento," Tesis, Universidad de Valladolid, 2020.

\bibitem{vinod} % Modelado y simulación de sistemas fotovoltaicos
R. Vinod, R. Kumar, and S. K. Singh, ``Solar photovoltaic modeling and simulation: As a renewable energy solution," \emph{Energy Reports}, vol. 4, Nov. 2018, pp. 701–712.

\bibitem{obeidat} % Revisión de sistemas fotovoltaicos futuros
F. Obeidat, ``A comprehensive review of future photovoltaic systems," \emph{Solar Energy}, vol. 163, Mar. 2018, pp. 545–559.

\bibitem{kipp} % Resumen de las normas internacionales de energía solar y la norma ISO 9060:2018
Kipp \& Zonen, ``Normas internacionales de energía solar,'' 2024. [En línea]. Disponible: \url{https://www.termodinamica.cl/cms/documents/2024_WQS_Normas-Internacionais-de-Energia-Solar_ESP.pdf}. Accessed: Oct. 12, 2025.

\bibitem{ordoñez} % Estudio que utiliza métricas estádisticas para evaluar el desempeño de ML para predecir recursos solares a partir de imagenes satelitales
L.E. Ordoñez Palacios, V. Bucheli Guerrero and H. Ordoñez, '"Machine Learning for Solar Resource Assessment Using Satellite Images," Energies (Basel), vol. 15, no. 11, Jun 1, pp. 3985.

\bibitem{balanzategui} % Estudio sobre la incertidumbre en los procesos de calibración de piranómetros
J.L. Balenzategui, M. Molero, J.P. Silva, F. Fabero, J. Cuenca, E. Mejuto and J. De Lucas, '"Uncertainty in the Calibration Transfer of Solar Irradiance Scale: From Absolute Cavity Radiometers to Standard Pyrheliometers," Solar, vol. 2, no. 2, Apr 2, pp. 158–185.

\bibitem{ogundimu} % Diseño de medidor de irradiancia y temperatura de bajo costo
E. Ogundimu, E. Akinlabi, C. Mgbemene, and I. Jacobs, ``Design and Implementation of a Low-cost Irradiance–Temperature Data Logging Meter for Solar PV Applications," \emph{African Journal of Mechanical and Industrial Engineering}, vol. 6, no. 4, Sept. 2019.

\bibitem{chouay} % Sensor neuronal en tiempo real para irradiancia solar
Y. Chouay and M. Ouassaid, ``An accurate real-time neural network based irradiance and temperature sensor for photovoltaic applications," \emph{Results in Engineering}, vol. 21, Jan. 2024.

\bibitem{altaani} % Medición de irradiancia solar con dispositivos inteligentes
H. Al-Taani and S. Arabasi, ``Solar Irradiance Measurements Using Smart Devices: A Cost-Effective Technique for Estimation of Solar Irradiance for Sustainable Energy Systems," \emph{Sustainability}, vol. 10, no. 2, Feb. 2018.

\bibitem{ansong} %Método de pronóstico de irradiancia utilizando sky-imagers
M. Ansong, G. Huang, T.N. Nyang’onda, R.J. Musembi and B.S. Richards, ``Very short-term solar irradiance forecasting based on open-source low-cost sky imager and hybrid deep-learning techniques," Solar energy, vol. 294, Jul 1, pp. 113516.

\bibitem{mercier} % Uso de transformadores de visión para medir irradiancia solar
T. M. Mercier, A. Sabet, and T. Rahman, ``Vision transformer models to measure solar irradiance using sky images in temperate climates," \emph{Applied Energy}, vol. 362, Mar. 2024.

\bibitem{sanchez} % Estimación de componentes de irradiancia con imágenes del cielo
C. D. Sánchez-Segura, L. Valentín-Coronado, M. I. Peña-Cruz, A. Díaz-Ponce, D. Moctezuma, G. Flores, and D. Riveros-Rosas, ``Solar irradiance components estimation based on a low-cost sky-imager," \emph{Solar Energy}, vol. 220, Apr. 2021, pp. 269–280.

\bibitem{roy} % Sistema embebido para medición de irradiancia solar
S. Roy, S. C. Panja, and S. N. Patra, ``An embedded system to measure ground-based solar irradiance signal," \emph{Measurement}, vol. 173, Oct. 2020.

\bibitem{kester} % Manual técnico sobre conversión de datos
W. Kester, ``Data Conversion Handbook," 1st ed., 2005.

\bibitem{rehpade} % Estudio sobre elección de capacitancia de desacople para mejorar la integridad de señal
R. Rehpade, G. Kharate and J. Chopade, ``Analysis of Decoupling Capacitor Performance in Improving Power Integrity in Two Layer/Three Layer Printed Circuit Boards," EJECE, vol. 7, no. 6, -11-20, pp. 34.

\bibitem{guangzhao} % Estudio sobre optimización de capacitores de desacople
G. Li, L. Zhai, H. Feng and H. Gu, ``Optimization Design Method of Decoupling Capacitor in PCB Hardware of Electric Vehicle Controller," Energy procedia, vol. 105, May, pp. 3201–3206.

\bibitem{ixolartm_datasheet} % Datasheet ANYSOLAR IXOLARTM SolarBIT KXOB25-14X1FTR-ND
Anysolar, ``\textit{-Datasheet-} ANYSOLAR IXOLARTM SolarBIT KXOB25-14X1FTR-ND,'' 2024. [En línea]. Disponible: \url{https://waf-e.dubudisk.com/anysolar.dubuplus.com/techsupport@anysolar.biz/O18Ae08/DubuDisk/www/Gen3/KXOB25-14X1F%20DATA%20SHEET%20202007.pdf}. Accessed: Oct. 12, 2025.

\bibitem{ina226_datasheet} % Texas Instruments INA226 Module
Texas Instruments, ``\textit{-Datasheet-} INA226 36V, 16-Bit, Ultra-Precise I2C Output Current, Voltage, and Power Monitor With Alert,'' 2024. [En línea]. Disponible: \url{https://www.ti.com/product/INA226?keyMatch=INA226&tisearch=universal_search&usecase=GPN-ALT#tech-docs}. Accessed: Oct. 12, 2025.

\bibitem{sx1276} % Semtech LoRa Module SX1276
Semtech Corporation, ``SX1276/77/78/79 LoRa Modem,'' 2019. [En línea]. Disponible: \url{https://www.semtech.com/products/wireless-rf/lora-connect/sx1276}. Accedido: Oct. 21, 2025.

\bibitem{lorawan}
LoRa Alliance, ``LoRaWAN 1.0.4 Specification,'' 2020. [En línea]. Disponible: \url{https://lora-alliance.org/resource_hub/lorawan-104-specification/}. Accedido: Oct. 21, 2025.

\bibitem{arentio_buy} % Página de compra de la marca Arentio
Arentio, ``\textit{-Buy-} Sensor de irradiación,'' [En línea]. Disponible:  \url{https://shop.arentio.com/shop/product/sensor-de-irradiacion-847}. Accessed: Oct. 12, 2025.

\bibitem{hukseflux_buy} % Página de compra de la marca Hukseflux
Hukseflux, ``\textit{-Buy-} Digital Class A pyranometer SR20-D2,'' [En línea]. Disponible: \url{https://www.hukseflux.com/products/pyranometers-solar-radiation-sensors/pyranometers/sr20-d2-pyranometer}. Accessed: Oct. 12, 2025.

\bibitem{bulk_buy} % Página de compra de capacitores TAJB476K010RNJ
DBU Electronics, ``\textit{-Buy-} Capacitor Tantalio SMD, 47uF x 10V,'' [En línea]. Disponible:  \url{https://www.dbuelectronics.cr/tantalio/221-capacitor-tantalio-smd-47uf-x-10v.html}. Accessed: Nov. 03, 2025.

\bibitem{cap_ceram_buy} % Página de compra de capacitores GCM188R71C104KA37J
Mouser Electronics, ``\textit{-Buy-} Capacitores cerámicos de capas múltiples (MLCC),'' [En línea]. Disponible:  \url{https://mou.sr/3Lnpw2H}. Accessed: Nov. 03, 2025.

\bibitem{ttgo_buy} % Página de compra de Lilygo para TTGO LoRa V2
LilyGo, ``\textit{-Buy-} TTGO LoRa32 V2.1\_1.6,'' [En línea]. Disponible: \url{https://lilygo.cc/products/lora3?variant=44778890494133}. Accessed: Nov. 03, 2025.

\bibitem{ixolartm_buy} % Página de compra de Digikey para las células solares
Digikey, ``\textit{-Buy-} ANYSOLAR IXOLARTM SolarBIT KXOB25-14X1FTR-ND,'' [En línea]. Disponible: \url{https://www.digikey.com/short/8rwp5h93}. Accessed: Nov. 04, 2025.

\bibitem{ups10982_buy} % Página de compra de Mouser para los termistores
Mouser Electronics, ``\textit{-Buy-} Termistores de coeficiente de temperatura negativo (NTC),'' [En línea]. Disponible: \url{https://mou.sr/3JMvIRf}. Accessed: Nov. 04, 2025.

\bibitem{ups10982_datasheet} % Datasheet Littlefuse UPS109**
Littlefuse, ``\textit{-Datasheet-} Thermistor Probes and Assemblies,''. [En línea]. Disponible: \url{https://www.littelfuse.com/assetdocs/littelfuse-thermistor-probes-assemblies-usp10972-datasheet?assetguid=90f74d46-daef-48f8-9bef-7d0f810f11ca}. Accessed: Nov. 04, 2025.

\bibitem{ina226_buy} % Página de compra de Amazon para los INA226
Amazon, ``\textit{-Buy-} INA226 36V, 16-Bit, Ultra-Precise I2C Output Current, Voltage, and Power Monitor With Alert,'' [En línea]. Disponible: \url{https://a.co/d/av6tTip}. Accessed: Nov. 04, 2025.

\bibitem{resistor.5_buy} % Página de compra de Mouser para los Resistencias SMD 0.5 Ohms
Mouser Electronics, ``\textit{-Buy-} Thin Film Resistors - SMD 0.5Ohms 2512 2W 1\%,'' [En línea]. Disponible: \url{https://mou.sr/47vcFTe}. Accessed: Nov. 04, 2025.

\bibitem{ads1115_datasheet} % Datasheet Texas Instruments ADS1115
Texas Instruments, ``\textit{-Datasheet-} ADS111x Ultra-Small, Low-Power, I2C-Compatible,''. [En línea]. Disponible: \url{https://www.ti.com/lit/ds/symlink/ads1114.pdf}. Accessed: Nov. 04, 2025.

\bibitem{ads1115_buy} % Página de compra de Crcibernética para los ADS1115
CrCibernética, ``\textit{-Buy-} ADS1115 4 Channel 16-Bit ADC with i2c,'' [En línea]. Disponible: \url{https://www.crcibernetica.com/ads1115-4-channel-16-bit-adc-with-i2c}. Accessed: Nov. 04, 2025.

\bibitem{parida} % Revisión general de tecnologías fotovoltaicas
B. Parida, S. Iniyan, and R. Goic, vA review of solar photovoltaic technologies," \emph{Renewable and Sustainable Energy Reviews}, vol. 15, no. 3, Jan. 2011, pp. 1625–1636.

\end{thebibliography}


% -------------------------------------------------------------------
% APÉNDICES
% -------------------------------------------------------------------

\appendices

\section{Glosario, abreviaturas y simbología}
\section{Manual de usuario}
\subsection{Procesos de Calibración}
\section{Información sobre la organización auspiciadora}
\subsection{Laboratorio de Sistemas Electrónicos para la Sostenibilidad}

% -------------------------------------------------------------------
% ANEXOS
% -------------------------------------------------------------------

\section*{Anexos}
\addcontentsline{toc}{section}{Anexos}
\subsection*{Anexo B.3 Esquemáticos eléctricos}
\subsection*{Anexo B.1 Hoja de datos KXOB25-14X1F-TR}
\subsection*{Anexo B.1 Hoja de datos INA226}
\subsection*{Anexo B.1 Hoja de datos ADS1115}
\subsection*{Anexo B.2 Hoja de datos LoRa SX1276}

\end{document}